\clearpage

\setcounter{thefigs}{0}

\clearpage
\stepcounter{thefigs}
\begin{figure}
\figurenum{\fignum}
\plotone{response_sdss.ps}
\caption{\label{response_sdss} Estimated filter response for all five
bands in the SDSS, as a function of rest frame wavelength for galaxies
at different redshifts. A 4 gigayear old instantaneous burst using the
models of \citet{bruzual93a} is shown for reference. {\it Top panel:}
The system consisting of \band{0.0}{u}, \band{0.0}{g}, \band{0.0}{r},
\band{0.0}{i}, and \band{0.0}{z}.  {\it Bottom panel:} The system
consisting of \band{0.0}{u}, \band{0.1}{g}, \band{0.1}{r},
\band{0.1}{i}, and \band{0.2}{z}. We use this second system rather
than the first because it requires less interpolation to determine
\band{0.1}{g}, \band{0.1}{r}, and \band{0.1}{i}, and no extrapolation to
determine \band{0.0}{u} and \band{0.1}{z}. }
\end{figure}

\clearpage
\stepcounter{thefigs}
\begin{figure}
\figurenum{\fignum}
%\plotone{plotaij.sample8b15.ps}
\caption{\label{plotaij.sample8b15} {\it Top panel}: Distribution of
the components of the three-parameter fit to the five-band SDSS
photometry for a random subsample consisting of 16,000 of the SDSS
galaxies. $a_0$ is linearly proportional to the flux between $3500\AA$
and $7000\AA$, while $a_1$ and $a_2$ contribute no flux in this
range. Thus, the ratios $a_1/a_0$ and $a_2/a_0$ describe the spectral
type of the galaxy. $a_2/a_0$ is the more variable parameter and thus
is the better separator of galaxy type. {\it Bottom panel}: At fixed
$a_1/a_0=-0.2$, the inferred spectra corresponding to various values
of $a_2/a_0$. Near $a_2/a_0=0.5$, the spectrum is similar to that of
an elliptical galaxy. For higher values, the spectrum becomes bluer. }
\end{figure}

\clearpage
\stepcounter{thefigs}
\begin{figure}
\figurenum{\fignum}
%\plotone{model.ps}
\caption{\label{model} Galaxy fluxes determined from the inferred
galaxy spectra (see Figure \ref{plotaij.sample8b15}) relative to the
observed galaxy fluxes, for all five SDSS bands, shown for a random
subsample consisting of 16,000 of the SDSS galaxies. The
residuals are shown against redshift and absolute magnitude, the two
most important systematic variables in the analysis of this paper. The
solid lines show the median values. The dashed lines show the 10\% and
90\% levels (corresponding to $1.3\sigma$ if the distribution were
Gaussian). The systematics are very small in $g$, $r$, $i$ and $z$,
but more significant in $u$. The scatter is largest in $u$ and
$z$. Nevertheless, the inferred fluxes are a good representation of
the observed fluxes.  }
\end{figure}

\clearpage
\stepcounter{thefigs}
\begin{figure}
\figurenum{\fignum}
%\plotone{colors.sample8b15.ps}
\caption{\label{colors.sample8b15} For galaxies in the absolute
magnitude range $-21.5<M_r<-21$, the color distribution calculated
from the inferred SEDs as a function of redshift. The left column
shows the four colors relative to \band{0.1}{r} for each galaxy versus
redshift. The right column shows the distribution of colors for
galaxies in the range $0.04<z<0.12$ {\it (dotted)} compared to the
distribution of colors of galaxies in the range $0.12<z<0.19$ {\it
(solid)}. The resulting color distribution is clearly fairly constant
with redshift.}
\end{figure}

\clearpage
\stepcounter{thefigs}
\begin{figure}
\figurenum{\fignum}
%\plotone{kcorrect.sample8b15.ps}
\caption{\label{kcorrect.sample8b15} $K$-corrections as a function of
redshift in all five bands for a random subsample consisting of 8,000
of the SDSS galaxies. The $K$-corrections are largest, and therefore
the most uncertain, for the \band{0.0}{u} and \band{0.1}{g} bands. The
range of $K$-corrections at each redshift reflects the range of galaxy
types at each redshift.}
\end{figure}

\clearpage
\stepcounter{thefigs}
\begin{figure}
\figurenum{\fignum}
%\plotone{ciCompare.sample8b15.ps}
\caption{\label{ciCompare.sample8b15} Difference in the
$K$-corrections in each band between the method used in Figure
\ref{kcorrect.sample8b15} and the method of simply interpolating
between adjacent bands fitting a power-law SED. The differences are
small in \band{0.1}{r}, \band{0.1}{i}, and \band{0.2}{z}, where galaxy
SEDs have simple shapes. There are large differences in \band{0.0}{u}
and \band{0.1}{g}, for which the 4000 \AA\ break is important. Note
particularly the systematic differences in \band{0.1}{g} with
redshift.}
\end{figure}

\clearpage
\stepcounter{thefigs}
\begin{figure}
\figurenum{\fignum}
%\plotone{cibreakCompare.sample8b15.ps}
\caption{\label{cibreakCompare.sample8b15} Same as Figure
\ref{ciCompare.sample8b15}, now comparing the $K$-corrections of Figure
\ref{kcorrect.sample8b15} with the ``interpolation with a break''
method. This method fits a power law between adjacent bands, except at
4000 \AA, where we fit for the size of the 4000\AA\ break (assuming
that $f(\lambda)\propto \lambda^2$ for $\lambda<4000$ \AA). This
greatly improves the agreement in \band{0.1}{g} while making the
disagreement in \band{0.0}{u} only slightly worse. These results
indicate that it important to account for the structure in the blue
region of the spectrum when performing $K$-corrections.}
\end{figure}

\clearpage
\stepcounter{thefigs}
\begin{figure}
\figurenum{\fignum}
%\plotone{specK.ps}
\caption{\label{specK} Similar to Figure \ref{ciCompare.sample8b15},
now comparing the $K$-corrections of Figure \ref{kcorrect.sample8b15}
with $K$-corrections determined from the spectra (which can only be
calculated for the \band{0.1}{g}, \band{0.1}{r}, and \band{0.1}{i}
bands). In all bands, the $K$-corrections are very similar, giving us
confidence in our results. This result is actually remarkable more for
what it says about the high quality of the spectrophotometry in the
SDSS survey.}
\end{figure}

\clearpage
\stepcounter{thefigs}
\begin{figure}
\figurenum{\fignum}
%\plotone{aperturevsz.M.sample8b15.ps}
\caption{\label{aperturevsz.M.sample8b15} }
\end{figure}

