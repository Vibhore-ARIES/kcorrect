\clearpage

\setcounter{thefigs}{0}

\clearpage
\stepcounter{thefigs}
\begin{figure}
\figurenum{\fignum}
\plotone{response_sdss.ps}
\caption{\label{response_sdss} Estimated filter response for all five
bands in the SDSS, as a function of observed wavelength. A 4 gigayear
old instantaneous burst using the models of \citet{bruzual93a} (and
observed at $z=0$) is shown for reference.}
%at different redshifts. A 4 gigayear old instantaneous burst using the
%models of \citet{bruzual93a} is shown for reference. {\it Top panel:}
%The system consisting of \band{0.0}{u}, \band{0.0}{g}, \band{0.0}{r},
%\band{0.0}{i}, and \band{0.0}{z}.  {\it Bottom panel:} The system
%consisting of \band{0.0}{u}, \band{0.1}{g}, \band{0.1}{r},
%\band{0.1}{i}, and \band{0.2}{z}. We use this second system rather
%than the first because it requires less interpolation to determine
%\band{0.1}{g}, \band{0.1}{r}, and \band{0.1}{i}, and no extrapolation to
%determine \band{0.0}{u} and \band{0.1}{z}. 
\end{figure}

%\clearpage
%\stepcounter{thefigs}
%\begin{figure}
%\figurenum{\fignum}
%%\plotone{k_espec_plot.ps}
%\caption{\label{k_espec_plot} The four derived eigenspectra. Note that
%eigenspectra \#1, \#2, and \#3 are constrained to have zero total flux in the
%range between 3500\AA and 7500\AA. Eigenspectrum \#0 is not in any
%sense the ``average'' spectrum. }
%\end{figure}

\clearpage
\stepcounter{thefigs}
\begin{figure}
\figurenum{\fignum}
\plotone{k_coeffdist_plot.ps}
\caption{\label{k_coeffdist_plot} {\it Top panels}: Distribution of
the components of the four-parameter fit to the five-band SDSS
photometry for a random subsample consisting of 10,000 of the SDSS
galaxies. $a_0$ is linearly proportional to the flux between $3500\AA$
and $7000\AA$, while $a_1$, $a_2$, and $a_3$ contribute no flux in
this range. Thus, the ratios $a_1/a_0$, $a_2/a_0$, and $a_3/a_0$
describe the spectral type of the galaxy. $a_1/a_0$ is the most
variable parameter and thus is the better separator of galaxy
type. {\it Bottom panel}: At fixed $a_2/a_0$ and $a_3/a_0$, the
inferred spectra corresponding to various values of $a_3/a_0$. Near
$a_3/a_0=-0.20$, the spectrum is similar to that of an elliptical
galaxy. For higher values, the spectrum becomes bluer. }
\end{figure}

\clearpage
\stepcounter{thefigs}
\begin{figure}
\figurenum{\fignum}
\plotone{k_model_plot.ps}
\caption{\label{k_model_plot} Reconstructed galaxy fluxes relative to
the observed galaxy fluxes, for all five SDSS bands, shown for a
random subsample consisting of around 10,000 of the SDSS galaxies. The
residuals are shown against redshift.  There is no systematic trend
with redshift in any band. The 5-$\sigma$ clipped estimate of the
scatter around the observed fluxes is listed for each band. In $u$,
$g$, $r$, and $i$ the scatter is consistent with the expected
photometric errors in the survey at all redshifts. At high redshift
the scatter in $z$ becomes large, most likely due to increasing
photometric errors. }
\end{figure}

\clearpage
\stepcounter{thefigs}
\begin{figure}
\figurenum{\fignum}
\plotone{main_colors_plot.ps}
\caption{\label{main_colors_plot} Color distributions $K$-corrected to
$z=0.1$ for Main Sample galaxies in the luminosity range
$-21.5<M_{\band{0.1}{r}}<-21$. This sample is complete for
$0.05<z<0.15$. Left panels show the colors as a function of
redshift. Right panel shows the distributions of each color at high
and low redshift.}
\end{figure}

\clearpage
\stepcounter{thefigs}
\begin{figure}
\figurenum{\fignum}
\plotone{lrg_colors_plot.ps}
\caption{\label{lrg_colors_plot} Same as Figure
\ref{main_colors_plot}, now $K$-corrected to $z=0.3$ for LRG galaxies
in the luminosity range $-22.8<M_{\band{0.3}{r}}<-22.5$. }
\end{figure}

\clearpage
\stepcounter{thefigs}
\begin{figure}
\figurenum{\fignum}
\plotone{spur.ps}
\caption{\label{spur} }
\end{figure}

\clearpage
\stepcounter{thefigs}
\begin{figure}
\figurenum{\fignum}
\plotone{k_kcorrect_plot.ps}
\caption{\label{kcorrect.sample8b15} $K$-corrections as a function of
redshift in all five bands for a random subsample consisting of 8,000
of the SDSS galaxies. The $K$-corrections are largest, and therefore
the most uncertain, for the \band{0.0}{u} and \band{0.1}{g} bands. The
range of $K$-corrections at each redshift reflects the range of galaxy
types at each redshift.}
\end{figure}

\clearpage
\stepcounter{thefigs}
\begin{figure}
\figurenum{\fignum}
\plotone{compareci.ps}
\caption{\label{compareci} Difference in the
$K$-corrections in each band between the method used in Figure
\ref{k_kcorrect_plot} and the method of simply interpolating
between adjacent bands fitting a power-law SED. The differences are
small in \band{0.1}{r}, \band{0.1}{i}, and \band{0.2}{z}, where galaxy
SEDs have simple shapes. There are large differences in \band{0.0}{u}
and \band{0.1}{g}, for which the 4000 \AA\ break is important. Note
particularly the systematic differences in \band{0.1}{g} with
redshift.}
\end{figure}


\clearpage
\stepcounter{thefigs}
\begin{figure}
\figurenum{\fignum}
\plotone{k_speck_plot.fiber.0.1.ps}
\caption{\label{k_speck_plot.fiber.0.1} }
\end{figure}

\clearpage
\stepcounter{thefigs}
\begin{figure}
\figurenum{\fignum}
\plotone{k_speck_plot.0.3.ps}
\caption{\label{k_speck_plot.0.3} }
\end{figure}
