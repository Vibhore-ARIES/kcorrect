\clearpage

\setcounter{thefigs}{0}

\clearpage
\stepcounter{thefigs}
\begin{figure}
\figurenum{\fignum}
\plotone{response_sdss.ps}
\caption{\label{response_sdss} Estimated response for all five bands
in the SDSS (accounting for the atmosphere, the mirrors, the filters,
and the CCDs), as a function of observed wavelength, as measured by
Mamoru Doi. The predicted spectrum $f(\lambda)$ for a 4 Gyr old
instantaneous burst using the models of \citet{bruzual93a} (and
observed at $z=0$) is shown for reference.}
%at different redshifts. A 4 Gyear old instantaneous burst using the
%models of \citet{bruzual93a} is shown for reference. {\it Top panel:}
%The system consisting of \band{0.0}{u}, \band{0.0}{g}, \band{0.0}{r},
%\band{0.0}{i}, and \band{0.0}{z}.  {\it Bottom panel:} The system
%consisting of \band{0.0}{u}, \band{0.1}{g}, \band{0.1}{r},
%\band{0.1}{i}, and \band{0.2}{z}. We use this second system rather
%than the first because it requires less interpolation to determine
%\band{0.1}{g}, \band{0.1}{r}, and \band{0.1}{i}, and no extrapolation to
%determine \band{0.0}{u} and \band{0.1}{z}. 
\end{figure}

\clearpage
\stepcounter{thefigs}
\begin{figure}
\figurenum{\fignum}
\plotone{spur.ps}
\caption{\label{spur} The spectrum corresponding to the direction of
the spur in the upper right panel of Figure
\ref{k_coeffdist_plot}. Note the strong feature near 4000
\AA. Overplotted are the bandpasses for $\band{0.3}{g}$ and
$\band{0.3}{r}$. The strong features fall in the gap between the
bandpasses. Thus, in a linear fit to a galaxy at $z=0.3$, this
component can be used to fit the observed magnitudes without being
constrained to have reasonable behavior around 4000 \AA. }
\end{figure}

%\clearpage
%\stepcounter{thefigs}
%\begin{figure}
%\figurenum{\fignum}
%%\plotone{k_espec_plot.ps}
%\caption{\label{k_espec_plot} The four derived eigenspectra. Note that
%eigenspectra \#1, \#2, and \#3 are constrained to have zero total flux in the
%range between 3500\AA and 7500\AA. Eigenspectrum \#0 is not in any
%sense the ``average'' spectrum. }
%\end{figure}

\clearpage
\stepcounter{thefigs}
\begin{figure}
\figurenum{\fignum}
\plotone{k_coeffdist_plot.ps}
\caption{\label{k_coeffdist_plot} {\it Top panels}: Distribution of
the components of the four-parameter fit to the five-band SDSS
photometry for a random subsample consisting of 10,000 of the SDSS
galaxies. $a_0$ is linearly proportional to the flux between 3500 \AA
and 7500 \AA, while $a_1$, $a_2$, and $a_3$ contribute no flux in
this range. Thus, the ratios $a_1/a_0$, $a_2/a_0$, and $a_3/a_0$
describe the spectral type of the galaxy. $a_3/a_0$ is the most
variable parameter and thus is the best separator of galaxy
type. The spur extending from the lower left to the upper right from
the red dot in the $(a_3/a_0)$-$(a_2/a_0)$ plane is due to a
degeneracy for galaxies at $z\sim 0.3$, described in detail in Section
\ref{grgap}. {\it Bottom panel}: At fixed $a_1/a_0$ and $a_2/a_0$, the
inferred spectra corresponding to various values of $a_3/a_0$. Near
$a_3/a_0=-0.20$, the spectrum is similar to that of an elliptical
galaxy. For higher values, the spectrum becomes bluer. }
\end{figure}

\clearpage
\stepcounter{thefigs}
\begin{figure}
\figurenum{\fignum}
\plotone{k_model_plot.ps}
\caption{\label{k_model_plot} Reconstructed galaxy fluxes relative to
the observed galaxy fluxes, for all five SDSS bands, shown for a
random subsample consisting of around 10,000 of the SDSS galaxies. The
residuals are shown against redshift.  There is no systematic trend
with redshift in any band. The 5-$\sigma$ clipped estimate of the
scatter around the observed fluxes is listed for each band. In $u$,
$g$, $r$, and $i$ the scatter is consistent with the expected
photometric errors in the survey at all redshifts. At high redshift
the scatter in $z$ becomes large, most likely due to increasing
photometric errors. }
\end{figure}

\clearpage
\stepcounter{thefigs}
\begin{figure}
\figurenum{\fignum}
\plotone{main_colors_plot.z.ps}
\caption{\label{main_colors_plot.z} Color distributions in the
observed frame for SDSS Main Sample galaxies in the luminosity range
$-21.5<M_{\band{0.1}{r}}<-21.2$. This sample is complete (that is,
volume limited) for $0.05<z<0.17$. Left panels show the colors as a
function of redshift. Right panel shows the distributions of each
color at high and low redshift within the volume-limited
subsample. The observed colors clearly depend strongly on redshift.}
\end{figure}

\clearpage
\stepcounter{thefigs}
\begin{figure}
\figurenum{\fignum}
\plotone{main_colors_plot.ps}
\caption{\label{main_colors_plot} Similar to Figure
\ref{main_colors_plot.z}, but now the colors are $K$-corrected to
$z=0.1$.  There is very little dependence of the colors on redshift,
even for the $\band{0.1}{u}$ band, where the low-redshift end is an
extrapolation of the data.}
\end{figure}

\clearpage
\stepcounter{thefigs}
\begin{figure}
\figurenum{\fignum}
\plotone{lrg_colors_plot.z.ps}
\caption{\label{lrg_colors_plot.z} Similar to Figure
\ref{main_colors_plot.z}, now showing LRG galaxies (Cut I) in the
luminosity range $-22.8<M_{\band{0.3}{r}}<-22.5$. Again, there is a
strong dependence on redshift.}
\end{figure}

\clearpage
\stepcounter{thefigs}
\begin{figure}
\figurenum{\fignum}
\plotone{lrg_colors_plot.ps}
\caption{\label{lrg_colors_plot} Same as Figure
\ref{lrg_colors_plot.z}, now $K$-correcting the LRG galaxies to
$z=0.3$.  The redshift dependence is greatly reduced for the LRGs in
comparison to Figure \ref{lrg_colors_plot.z}; on the other hand, there
are distinct trends of restframe color with redshift.  In
$\band{0.3}{(g-r)}$ an overall trend is apparent; LRGs at $z=0.4$ are
about 0.1 magnitudes bluer than LRGs at $z=0.2$. A change of this
magnitude is attributable to passive galaxy evolution, though
considerably more work needs to be done to show that this is
occurring. In $\band{0.3}{(r-i)}$, a blueward shift also occurs,
though at a much smaller level.}
\end{figure}

\clearpage
\stepcounter{thefigs}
\begin{figure}
\figurenum{\fignum}
\plotone{k_kcorrect_plot.ps}
\caption{\label{k_kcorrect_plot} $K$-corrections to $z=0.3$ as a
function of redshift in all five bands for a random subsample
consisting around 10,000 of the SDSS galaxies.  The range of
$K$-corrections at each redshift reflects the range of galaxy types at
each redshift.  The $K$-corrections are largest, and therefore the
most uncertain, for the \band{0.3}{u} and \band{0.3}{g} bands. While
we show the $K$-corrections for \band{0.3}{u} at $z<0.3$ and for
\band{0.3}{z} at $z>0.3$, and indeed these $K$-corrections are fairly
well-behaved, we do not recommend using these extrapolated results for
scientific purposes. }
\end{figure}

\clearpage
\stepcounter{thefigs}
\begin{figure}
\figurenum{\fignum}
\plotone{k_speck_plot.fitfib.0.1.ps}
\caption{\label{k_speck_plot.fitfib.0.1} Difference between the
$K$-corrections to $z=0.1$ determined from the spectroscopy and those
determined from the analysis of broad-band magnitudes {\it
synthesized} from the same spectra.}
\end{figure}

\clearpage
\stepcounter{thefigs}
\begin{figure}
\figurenum{\fignum}
\plotone{compareci.ps}
\caption{\label{compareci} Difference in the $K$-corrections to
$z=0.1$ in each band between the method used in Figure
\ref{k_kcorrect_plot} and the method of simply interpolating between
adjacent bands fitting a power-law SED. The differences are small in
\band{0.1}{r}, \band{0.1}{i}, and \band{0.1}{z}, where galaxy SEDs
have simple shapes. There are large systematic differences in
\band{0.1}{u} and \band{0.1}{g}, for which the 4000 \AA\ break is
important in the spectral templates used. }
\end{figure}

\clearpage
\stepcounter{thefigs}
\begin{figure}
\figurenum{\fignum}
\plotone{comparecibreak.ps}
\caption{\label{comparecibreak} Same as Figure \ref{compareci},
now comparing to a method of interpolating the bandpasses using
power-laws, and fitting for the 4000 \AA\ break. The systematic trends
in \band{0.1}{u} and \band{0.1}{g} are gone (though there is
considerable scatter in \band{0.1}{u}). }
\end{figure}

\clearpage
\stepcounter{thefigs}
\begin{figure}
\figurenum{\fignum}
\plotone{k_speck_plot.0.1.ps}
\caption{\label{k_speck_plot.0.1} Difference between the
$K$-corrections to $z=0.1$ determined from the spectroscopy and those
determined from the analysis of the broad-band Petrosian magnitudes, for
Main Sample galaxies.}
\end{figure}

\clearpage
\stepcounter{thefigs}
\begin{figure}
\figurenum{\fignum}
\plotone{k_speck_plot.0.3.ps}
\caption{\label{k_speck_plot.0.3} Difference between the
$K$-corrections to $z=0.3$ determined from the spectroscopy and those
determined from the analysis of the broad-band model magnitudes, for
LRGs.}
\end{figure}
