\clearpage

\setcounter{thefigs}{0}

\clearpage
\stepcounter{thefigs}
\begin{figure}
\figurenum{\fignum}
\plotone{k_sfh.ps}
\caption{\label{sfh} The top panel are the star-formation histories of
the 15 bursts we use in our model. The bottom panel shows (for no dust
and for $Z=0.008$) the PEGASE2 spectrum predicted for these bursts. }
\end{figure}

\clearpage
\stepcounter{thefigs}
\begin{figure}
\figurenum{\fignum}
\plotone{gmrz.ps}
\caption{\label{gmrz} }
\end{figure}

\clearpage
\stepcounter{thefigs}
\begin{figure}
\figurenum{\fignum}
\plotone{kfit.ps}
\caption{\label{kfit} The color residuals between the observed values
  and those of the best fit to the nonnegative combination of 225
  templates ($\delta(b-r)\equiv [b-r]_{\mathrm{obs}} -
  [b-r]_{\mathrm{fit}}$). The greyscale is proportional to the
  distribution of color residual at a given redshift. The lines are
  0.16, 0.45, and 0.84 fractional quantiles. Note that we have zoomed
  into $g-r$, $r-i$, and $i-z$ in order to see the residuals
  better. At high redshift many galaxies have abnormally low $u$-band
  fluxes relative to their $r$-band flux. At low redshift there are a
  few galaxies whose near infrared $z$-band magnitudes are
  lower. There is a general offset of nearly 0.1 magnitudes in
  $J-r$. }
\end{figure}
