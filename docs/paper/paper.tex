\documentclass[10pt,preprint]{aastex}

\newcommand{\vv}[1]{{\bf #1}}
\newcommand{\df}{\delta}
\newcommand{\dfft}{{\tilde{\delta}}}
\newcommand{\betaft}{{\tilde{\beta}}}
\newcommand{\erf}{{\mathrm{erf}}}
\newcommand{\erfc}{{\mathrm{erfc}}}
\newcommand{\Step}{{\mathrm{Step}}}
\newcommand{\ee}[1]{\times 10^{#1}}
\newcommand{\avg}[1]{{\langle{#1}\rangle}}
\newcommand{\Avg}[1]{{\left\langle{#1}\right\rangle}}
\def\simless{\mathbin{\lower 3pt\hbox
	{$\,\rlap{\raise 5pt\hbox{$\char'074$}}\mathchar"7218\,$}}} % < or of order
\def\simgreat{\mathbin{\lower 3pt\hbox
	{$\,\rlap{\raise 5pt\hbox{$\char'076$}}\mathchar"7218\,$}}} % > or of order
\newcommand{\iras}{{\sl IRAS\/}}
\newcommand{\petroratio}{{{\mathcal{R}}_P}}
\newcommand{\petroradius}{{{r}_P}}
\newcommand{\petronumber}{{{N}_P}}
\newcommand{\petroratiolim}{{{\mathcal{R}}_{P,\mathrm{lim}}}}
\newcommand{\band}[2]{\ensuremath{^{#1}{#2}}}

\setlength{\footnotesep}{9.6pt}

\newcounter{thefigs}
\newcommand{\fignum}{\arabic{thefigs}}

\newcounter{thetabs}
\newcommand{\tabnum}{\arabic{thetabs}}

\newcounter{address}

%% You can insert a short comment on the title page using the command below.

\slugcomment{Submitted to \aj}

%% If you wish, you may supply running head information, although
%% this information may be modified by the editorial offices.
%% The left head contains a list of authors,
%% usually a maximum of three (otherwise use et al.).  The right
%% head is a modified title of up to roughly 44 characters.  Running heads
%% will not print in the manuscript style.

\shortauthors{Blanton {\it et al.} (2000)}
\shorttitle{Modeling Galaxy SEDs}

%% This paper uses runs 752/756, rerun 4
%% It uses redshifts from version 3 of spectro1d

%% This is the end of the preamble.  Indicate the beginning of the
%% paper itself with \begin{document}.

\begin{document}
 
%% LaTeX will automatically break titles if they run longer than
%% one line. However, you may use \\ to force a line break if
%% you desire.

\title{Estimating Fixed-frame Galaxy Magnitudes in the SDSS
Spectroscopic Survey}


%% Use \author, \affil, and the \and command to format
%% author and affiliation information.
%% Note that \email has replaced the old \authoremail command
%% from AASTeX v4.0. You can use \email to mark an email address
%% anywhere in the paper, not just in the front matter.
%% As in the title, you can use \\ to force line breaks.

%\author{Michael Blanton}
%\affil{NASA/Fermilab Astrophysics Center\\
%Fermi National Accelerator Laboratory, Batavia, IL 60510-0500}
%\author{\and lots and lots of other, probably more important people}
%\affil{Some Institution\\
%Somewhere, Some City, Some State or Province}
%\email{blanton@fnal.gov}

% Authorship determined by those I was directly involved with
% in performing this work as well as those responsible for the photo-z
% plates and those who have characterized the filter curves (since
% these might be publicly distributed). 
\author{
Michael R. Blanton\altaffilmark{\ref{NYU}},
%Tamas Budavari\altaffilmark{\ref{JHU}},
%Andrew J. Connolly\altaffilmark{\ref{Pitt}},
Istv\'an Csabai\altaffilmark{\ref{JHU}},
Mamoru Doi\altaffilmark{\ref{Tokyo}},
Daniel Eisenstein\altaffilmark{\ref{Arizona}},
James E. Gunn\altaffilmark{\ref{Princeton}}, and
David W. Hogg\altaffilmark{\ref{NYU}}
%David J. Schlegel\altaffilmark{\ref{Princeton}}
%Julianne Dalcanton\altaffilmark{\ref{UW}},
%Jon Loveday\altaffilmark{\ref{Sussex}},
%Michael A. Strauss\altaffilmark{\ref{Princeton}},
%Mark SubbaRao\altaffilmark{\ref{Chicago}},
%David H. Weinberg\altaffilmark{\ref{Ohio}},
%John E. Anderson, Jr.\altaffilmark{\ref{Fermilab}},
%James Annis\altaffilmark{\ref{Fermilab}},
%Neta A. Bahcall\altaffilmark{\ref{Princeton}},
%Mariangela Bernardi\altaffilmark{\ref{Chicago}},
%J. Brinkmann\altaffilmark{\ref{APO}},
%Robert J. Brunner\altaffilmark{\ref{Caltech}},
%Scott Burles\altaffilmark{\ref{Fermilab}},
%Larry Carey\altaffilmark{\ref{UW}},
%Francisco J. Castander\altaffilmark{\ref{Chicago}, \ref{Pyrenees}},
%Andrew J. Connolly\altaffilmark{\ref{Pitt}},
%Istv\'an Csabai\altaffilmark{\ref{JHU}},
%Douglas Finkbeiner\altaffilmark{\ref{Berkeley}},
%Scott Friedman\altaffilmark{\ref{JHU}},
%Joshua A. Frieman\altaffilmark{\ref{Fermilab}},
%Masataka Fukugita\altaffilmark{\ref{CosmicRay},\ref{IAS}},
%G. S. Hennessy\altaffilmark{\ref{USNO}},
%Robert B. Hindsley\altaffilmark{\ref{USNO}},
%Takashi Ichikawa\altaffilmark{\ref{Tokyo}},
%\v{Z}eljko Ivezi\'{c}\altaffilmark{\ref{Princeton}},
%Stephen Kent\altaffilmark{\ref{Fermilab}},
%G. R.~Knapp\altaffilmark{\ref{Princeton}},
%D. Q.~Lamb\altaffilmark{\ref{Chicago}},
%R. French Leger\altaffilmark{\ref{UW}},
%Daniel C. Long\altaffilmark{\ref{APO}},
%Robert H. Lupton\altaffilmark{\ref{Princeton}},
%Timothy A.~McKay\altaffilmark{\ref{Michigan}},
%Avery Meiksin\altaffilmark{\ref{Edinburgh}},
%Aronne Merelli\altaffilmark{\ref{Caltech}},
%Jeffrey A. Munn\altaffilmark{\ref{USNO}},
%Vijay Narayanan\altaffilmark{\ref{Princeton}},
%Matt Newcomb\altaffilmark{\ref{CarnegieMellon}},
%R. C. Nichol\altaffilmark{\ref{CarnegieMellon}},
%Sadanori Okamura\altaffilmark{\ref{Tokyo}},
%Russell Owen\altaffilmark{\ref{UW}},
%Jeffrey R.~Pier\altaffilmark{\ref{USNO}},
%Adrian Pope\altaffilmark{\ref{JHU}},
%Marc Postman\altaffilmark{\ref{STScI}},
%Thomas Quinn\altaffilmark{\ref{UW}},
%Constance M. Rockosi\altaffilmark{\ref{Chicago}},
%Donald P. Schneider\altaffilmark{\ref{PennState}}, 
%Kazuhiro Shimasaku\altaffilmark{\ref{Tokyo}},
%Walter A. Siegmund\altaffilmark{\ref{UW}},
%Stephen Smee\altaffilmark{\ref{Maryland}},
%Yehuda Snir\altaffilmark{\ref{CarnegieMellon}},
%Chris Stoughton\altaffilmark{\ref{Fermilab}},
%Christopher Stubbs\altaffilmark{\ref{UW}},
%Alexander S.~Szalay\altaffilmark{\ref{JHU}},
%Gyula P.~Szokoly\altaffilmark{\ref{Potsdam}},
%Aniruddha R.~Thakar\altaffilmark{\ref{JHU}},
%Christy Tremonti\altaffilmark{\ref{JHU}},
%Douglas L. Tucker\altaffilmark{\ref{Fermilab}},
%Alan Uomoto\altaffilmark{\ref{JHU}},
%Dan vanden Berk\altaffilmark{\ref{Fermilab}},
%Michael S. Vogeley\altaffilmark{\ref{Drexel}},
%Patrick Waddell\altaffilmark{\ref{UW}},
%Brian Yanny\altaffilmark{\ref{Fermilab}},
%Naoki Yasuda\altaffilmark{\ref{NAOJ}},
%and Donald G.~York\altaffilmark{\ref{Chicago}}
}

\altaffiltext{1}{Based on observations obtained with the
Sloan Digital Sky Survey} 
\setcounter{address}{2}
\altaffiltext{\theaddress}{
\stepcounter{address}
New York University, Department of Physics, 4 Washington Place, New
York, NY 10003
\label{NYU}}
\altaffiltext{\theaddress}{
\stepcounter{address}
Department of Physics and Astronomy, The Johns Hopkins University,
Baltimore, MD 21218
\label{JHU}}
%\altaffiltext{\theaddress}{
%\stepcounter{address}
%University of Pittsburgh,
%Department of Physics and Astronomy,
%3941 O'Hara Street,
%Pittsburgh, PA 15260
%\label{Pitt}}
%\altaffiltext{\theaddress}{
%\stepcounter{address}
%rinceton University Observatory, Princeton,
%NJ 08544
%\label{Princeton}}
%\addtocounter{address}{1}
%\altaffiltext{\theaddress}{
%\stepcounter{address}
%Fermi National Accelerator Laboratory, P.O. Box 500,
%Batavia, IL 60510
%\label{Fermilab}}
%\altaffiltext{\theaddress}{
%\stepcounter{address}
%Department of Astronomy, University of Washington,
%Box 351580,
%Seattle, WA 98195 
%\label{UW}}
%\altaffiltext{\theaddress}{
%\stepcounter{address}
%University of Chicago, Astronomy \&
%Astrophysics Center, 5640 S. Ellis Ave., Chicago, IL 60637
%\label{Chicago}}
%\altaffiltext{\theaddress}{
%\stepcounter{address}
%Hubble Fellow 
%\label{Hubble}}
%\altaffiltext{\theaddress}{
%\stepcounter{address}
%Sussex Astronomy Centre,
%University of Sussex,
%Falmer, Brighton BN1 9QJ, UK
%\label{Sussex}}
%\altaffiltext{\theaddress}{
%\stepcounter{address}
%Ohio State University,
%Department of Astronomy,
%Columbus, OH 43210
%\label{Ohio}}
%\altaffiltext{\theaddress}{
%\stepcounter{address}
%Apache Point Observatory,
%2001 Apache Point Road,
%P.O. Box 59, Sunspot, NM 88349-0059
%\label{APO}}
%\altaffiltext{\theaddress}{
%\stepcounter{address}
%Department of Astronomy, California Institute of Technology,
%Pasadena, CA 91125
%\label{Caltech}}
%\altaffiltext{\theaddress}{
%\stepcounter{address}
%Observatoire Midi-Pyr\'en\'ees, 
%14 ave Edouard Belin, Toulouse, F-31400, France
%\label{Pyrenees}}
%\altaffiltext{\theaddress}{
%\stepcounter{address}
%Department of Astronomy and Research Center for 
%the Early Universe,
%School of Science, University of Tokyo,
%Tokyo 113-0033, Japan
%\label{Tokyo}}
%\altaffiltext{\theaddress}{
%\stepcounter{address}
%UC Berkeley, Dept. of Astronomy, 601 Campbell Hall, Berkeley, CA  94720-3411
%\label{Berkeley}}
%\altaffiltext{\theaddress}{
%\stepcounter{address}
%Institute for Cosmic Ray Research, University of
%Tokyo, Midori, Tanashi, Tokyo 188-8502, Japan
%\label{CosmicRay}}
%\altaffiltext{\theaddress}{
%\stepcounter{address}
%Institute for Advanced Study, Olden Lane,
%Princeton, NJ 08540
%\label{IAS}}
%\altaffiltext{\theaddress}{
%\stepcounter{address}
%U.S. Naval Observatory,
%3450 Massachusetts Ave., NW,
%Washington, DC  20392-5420
%\label{USNO}}
%\altaffiltext{\theaddress}{
%\stepcounter{address}
%University of Michigan, Department of Physics,
%500 East University, Ann Arbor, MI 48109
%\label{Michigan}}
%\altaffiltext{\theaddress}{
%\stepcounter{address}
%Department of Physics \& Astronomy,
%The University of Edinburgh,
%James Clerk Maxwell Building,
%The King's Buildings,
%Mayfield Road,
%Edinburgh EH9 3JZ, UK
%\label{Edinburgh}}
%\altaffiltext{\theaddress}{
%\stepcounter{address}
%Department of Physics, Carnegie Mellon University, 
%5000 Forbes Avenue, Pittsburgh, PA 15213-3890 
%\label{CarnegieMellon}}
%\altaffiltext{\theaddress}{
%\stepcounter{address}
%Space Telescope Science Institute, Baltimore, MD 21218
%\label{STScI}}
%\altaffiltext{\theaddress}{
%\stepcounter{address}
%Department of Astronomy and Astrophysics,
%The Pennsylvania State University,
%University Park, PA 16802
%\label{PennState}}
%\altaffiltext{\theaddress}{
%\stepcounter{address}
%Department of Astronomy,
%University of Maryland,
%College Park, MD 20742-2421 
%\label{Maryland}}
%\altaffiltext{\theaddress}{
%\stepcounter{address}
%Astrophysikalisches Institut Potsdam,
%An der Sternwarte 16, D-14482 Potsdam, Germany
%\label{Potsdam}}
%\altaffiltext{\theaddress}{
%\stepcounter{address}
%Department of Physics, Drexel University, Philadelphia, PA 19104
%\label{Drexel}}
%\altaffiltext{\theaddress}{
%\stepcounter{address}
%National Astronomical Observatory, Mitaka, Tokyo 181-8588, Japan
%\label{NAOJ}}
%\addtocounter{address}{1}
%\altaffiltext{\theaddress}{Physics Dept., University of California, Davis, CA 95616
%\label{UCDavis}}
%\addtocounter{address}{1}
%\altaffiltext{\theaddress}{IGPP/Lawrence Livermore National Laboratory
%\label{IGPP}}
%\addtocounter{address}{1}
%\altaffiltext{\theaddress}{Department of Astronomy, University of California, Berkeley, C
%A 94720-3411
%\label{Berkeley}}
%\stepcounter{address}
%\altaffiltext{\theaddress}{Remote Sensing Division, Code 7215, Naval
%Research Laboratory, Washington, DC 20375
%\label{NRL}}
%\addtocounter{address}{1}
%\altaffiltext{\theaddress}{U.S. Naval Observatory, Flagstaff Station,
%P.O. Box 1149,
%Flagstaff, AZ  86002-1149
%\label{Flagstaff}}

\clearpage

%% Mark off your abstract in the ``abstract'' environment. In the manuscript
%% style, abstract will output a Received/Accepted line after the
%% title and affiliation information. No date will appear since the author
%% does not have this information. The dates will be filled in by the
%% editorial office after submission.
\begin{abstract}
Broad-band measurements of galaxy fluxes suffer from the difficulty
that for galaxies at different redshifts the observations probe
different regions of the rest-frame galaxy spectrum. Certain
astronomical questions, such as the evolution of the luminosity
function of galaxies, require transforming these inherently
redshift-dependent magnitudes into redshift-independent quantities. To
prepare to address these astronomical questions, investigated in
detail in subsequent papers, we implement here the method of
\citet{csabai00a} for fitting spectral energy distributions (SEDs) to
broad-band photometric observations, in the context of the optical
observations of the Sloan Digital Sky Survey (SDSS). Linear
combinations of four spectral templates can reproduce the five SDSS
magnitudes of all galaxies to the precision of the
photometry. Expressed in the appropriate coordinate system, the
distribution of the coefficients multiplying the templates is planar,
and in fact nearly linear. The resulting reconstructed SEDs can be
used to recover fixed-frame magnitudes over a range of redshifts. We
show that the results of doing so yield consistent results, in the
sense that within each sample the intrinsic colors of similar type
galaxies are nearly constant with redshift. We note one remaining
problem associated with the gap between the $g$ and $r$ bands. We
compare our results to simpler interpolation methods and galaxy
spectrophotometry.  The software which generates these results is
publicly available and easily adapted to handle galaxy observations in
any AB magnitude system in the rest-frame optical regime.
\end{abstract}

\keywords{galaxies: fundamental parameters --- galaxies: photometry
--- galaxies: statistics}

%
% Introduction and motivation
%

\section{Motivation}
\label{motivation}

In the future, we expect that observations of galaxies (and indeed of
any astronomical sources) will be performed using integral field
spectrographs or equivalent devices which combine high spatial
resolution and high spectral resolution. By that time, the phenomena
of broad-band filters and the quaint terminology surrounding their
usage --- magnitudes, $K$-corrections, color terms, {\it etc.} ---
will have long since been forgotten. However, today, virtually all
observations of galaxies are made through such filters, and special
care has to be taken to recover knowledge of galaxy spectral energy
distributions (SEDs) from these observations.  Reconstructing galaxy
SEDs is nontrivial because SEDs of galaxies contain important
information on scales considerably smaller than the width of typical
broad-band filters. Furthermore, SEDs of distant galaxies are
redshifted such that the more distant the galaxy, the further the
observed bandpass is blueshifted relative to the rest-frame spectral
energy distribution of the object observed.  This paper focuses on a
method of facing these problems which (in the optical wavelength
regime) is fast, robust, and consistent over a large range of
redshifts.

In the past, people have in general accounted for these effects using
``$K$-corrections'' applied to the observed magnitudes. In these
analyses, a function $K(z)$ is added to the standard cosmological
bolometric distance modulus $\mathrm{DM}(z)$. Sometimes a single
function has been applied regardless of the galaxy type, though more
recently it has become standard to use a discrete set of $K(z)$
functions depending on galaxy type (based either on morphology or
spectral features). As described in \cite{oke68a} and later papers,
the function is based on the projection of an assumed galaxy SED
redshifted to $z$ onto the measured instrument response as a function
of wavelength (including the effects of the atmosphere, the
reflectivity of the mirrors, the filter transmission, and the response
of the CCD device or photographic emulsion).  However, galaxies do not
all have the same SED, nor are they selected from some discrete set of
SEDs. For this reason, these schemes for applying the $K$-corrections
can fail to be self-consistent: the SED assumed for the
$K$-corrections can be significantly inconsistent with the observed
galaxy colors! As we demand more precision from our astronomical data
analysis in new, large multi-band surveys, and in particular as we try
to quantify the evolution of galaxies, we must take a more
sophisticated approach to approximating fixed-frame observations of
galaxies.

Our approach here is to use a method for inferring the underlying SEDs
of a set of galaxies at a range of redshifts by requiring that their
SEDs all be drawn from a similar population. For each galaxy, we will
recover a model SED, which can be used to synthesize the galaxy's
magnitude in any bandpass. Although the operation we are performing on
the magnitudes is not strictly speaking a ``$K$-correction,'' we will
refer to it as such in this and subseqent papers. Our approach is
equivalent (nearly identical) to the photometric metric redshift
estimation methods of \citet{csabai00a} and \citet{budavari00a},
except we use slightly different coordinate systems. In fact, the
software includes a fast and relatively accurate photometric redshift
estimator based on their method.

We implemented this system and are publishing it in order to prepare
the reader for upcoming papers which will rely heavily on the
reliability of the fixed-frame magnitudes determined here.  In
addition, having implemented the fits, there is no reason not to share
them with the astronomical community; thus, this paper serves the
secondary purpose of describing the release of a piece of
software. The computer software we distribute builds itself into a C
shared object library, around which we have written both stand-alone C
programs and IDL routines. We have made the source code available
publicly, through a URL on the World Wide Web and through a public CVS
repository. Improvements or ports to other languages implemented by
users may be incorporated into the code upon request. The conditions
of use for the code are that this paper is cited in any resulting
refereed journal article and that the version of the code used is
specified in any such paper. The version of the code used to make the
figures for this paper is {\tt kcorrect v1\_2}.

Section \ref{sedfit} describes our method of fitting galaxy SEDs to
broad-band photometry. Section \ref{data} applies the method to
galaxies in the SDSS, showing that the fits are robust. Section
\ref{kcorrection} shows how one derives $K$-corrections from the
results, and discusses how best to use the results of such
fits. Section \ref{conclusions} concludes and discusses future
development of the method described here.

\section{Calculating Fixed-frame Galaxy Magnitudes}
\label{sedfit}

First, we describe how we reconstruct galaxy SEDs from broad-band
magnitude measurements. Second, we describe how we convert these SEDs
into estimates of fixed-frame galaxy magnitudes. 

\subsection{Fitting SEDs to Galaxy Broad-band Magnitudes}

Our task is to recover a model for the galaxy SED from broad-band
photometric measurements. Since a set of broad band galaxy fluxes does
not correspond uniquely to a particular SED, and because galaxy SEDs
are known to have significant structure over wavelength ranges small
compared to our bandpass, this task is an ill-posed, inverse
problem. However, we are not completely ignorant about the forms which
galaxy SEDs take, so we can attempt to use what we know about galaxy
SEDs to simplify our task.  The method described here for doing so
follows closely the methods of estimating photometric redshifts used
by \citet{csabai00a} and \citet{budavari00a}. It is designed to take
advantage of what we already understand about galaxy spectral energy
distributions.
%In addition, as implemented here, the method is
%equivalent to a fit to the star-formation history of the galaxies
%using a spectrophotometric model; we will investigate the implications
%of the model fits in a separate paper and concentrate here on the
%empirical aspects of the results.

Let us begin with an SED space defined by $N_b$ template galaxy SEDs
$\vv{v}_i(\lambda)$ (say from \citealt{bruzual93a}), where $N_b$ is
large. In principle, this should be a complete set of galaxy SEDs if
we want {\it all} galaxy SEDs to be in the space spanned by the
$\vv{v}_i(\lambda)$; in practice, this property is not necessary.
Given a set of $\vv{v}_i(\lambda)$, one can find an orthonormal set of
spectra $\vv{b}_i(\lambda)$ spanning the same space.  To be more
precise, we define the dot-product in the space of galaxy spectra to
be:
\begin{equation}
\vv{v}_i \cdot \vv{v}_j =
\int_{\lambda_{\mathrm{min}}}^{\lambda_{\mathrm{max}}} d\lambda
v_i(\lambda) v_j(\lambda),
\end{equation}
where $\lambda_{\mathrm{min}}$ and $\lambda_{\mathrm{max}}$ define the
wavelength range used to define orthogonality. These limits are
defined not over the full range covered by the $\vv{v}_i(\lambda)$
spectra, but instead over a smaller range which is within observable
wavelengths over most of the redshift range of the
sample. \footnote{Were we to define orthogonality over the full range
of $\vv{v}(\lambda)$, much of which is unobservable, the geometry of
the distribution of galaxies in this space would be more complicated.
If the template spectra were identical in the subrange chosen and only
differed outside of it, our choice would cause computational
problems, but in fact the spectra are all independent in the
restricted, as well as the full, wavelength range.}  We then define
the basis of the SED space such that the $N_b$ template SEDs,
$\vv{v}_i$, are linear combinations of $N_b$ basis spectra $\vv{b}_i$
and that $\vv{b}_i \cdot \vv{b}_j = \delta_{ij}$. These conditions do
not fully specify the $\vv{b}_i$, which naturally may be rotated
arbitrarily within our $N_b$-dimensional subspace of spectral space.

In our case the number of observed bands is less than $N_b$, so we
cannot determine an individual galaxy's position in this
$N_b$-dimensional space from its broad-band colors alone.  However, as
outlined by \citet{csabai00a}, one can find a lower dimensional space
defined by $N_t$ vectors $\vv{e}_k$ which best fits the {\it set} of
galaxies, as follows. The SED of galaxy $i$ may be reconstructed from a
linear combination of the basis spectra in this lower dimensional
space:
\begin{equation}
\label{lincomb}
f_{\mathrm{rec},i}(\lambda) d\lambda = 
\sum_k a_{i,k} \sum_j e_{k,j} b_j(\lambda) d\lambda,
\end{equation}
where $a_k$ are the components of the galaxy projected in the
low-dimensional space. From this SED it is easy to determine the
reconstructed flux
$F_{\mathrm{rec},il}$ in each survey bandpass by projecting
the model SED onto the bandpass $l$: 
\begin{equation}
F_{\mathrm{rec},il} = \int_0^\infty d\lambda
f_{\mathrm{rec},i}(\lambda) S_l(\lambda),
\end{equation}
where $S_l(\lambda)$ is the response of the instrument in band $l$.  We define
\begin{equation}
\chi^2 = \sum_i \sum_l
\frac{(F_{\mathrm{obs},il}-F_{\mathrm{rec},il})^2}{\sigma_{il}^2},
\end{equation}
where $\sigma_{il}$ is the estimated error in the observed flux
$F_{\mathrm{obs},il}$ in bandpass $l$ for galaxy $i$. Taking
derivatives of $\chi^2$ with respect to $a_{i,k}$ and $e_{k,j}$
results in a set of equations bilinear in $a_{i,k}$ and $e_{k,j}$.
\citet{csabai00a} describe how to iteratively solve this bilinear
equation by first fixing $e_{k,j}$ and fitting for $a_{i,k}$, then
fixing $a_{i,k}$ and fitting for $e_{k,j}$, and iterating that
procedure.

%We put one additional constraint on our SEDs. To regularize the edges
%of the SEDs, we define two additional bands blueward and redward of
%$u$ and $z$, respectively, which we will call $a_1$ and $a_2$. The
%bandpasses are simply Gaussians centered at 2700\AA\ and 10200\AA\
%with standard deviations of 250\AA\ and 350\AA, respectively. From
%studying synthetic galaxy SEDs redshifted between $z=0$ and $z=0.2$,
%it is clear that the $u-a_1$ color varies not much more than 1.5
%magnitudes around the value 1.5, and that the $z-a_2$ color usually
%varies only by about 0.2 magnitudes around 0.1. Thus, given the $u$
%and $z$ magnitude, we can put a very weak constraint on the $a_1$ and
%$a_2$ magnitudes.  Similarly, when one of the SDSS bands is
%unobserved, we replace it based on an adjacent band and the mean color
%for galaxies in the sample, with a large error bar. These weak
%constraints prevent the inferred galaxy SEDs from becoming too
%unreasonable in the regions of the spectrum without data.

We define the ``total flux'' $f_k$ of each template as the flux in the
range $\lambda_{\mathrm{min}}<\lambda<\lambda_{\mathrm{max}}$; then an
estimate of the flux in this range is $F_{{t}}=\Sigma_k a_k f_k$.
Galaxies of some fixed $F_t$ clearly lie on a plane in the component
space $a_k$. Therefore, a natural choice for the orientation of the
axes $\vv{e}_k$ is to let $\vv{e}_0$ be perpendicular to the planes of
constant flux; in this manner, the coefficient $a_0$ in Equation
(\ref{lincomb}) is directly proportional to $F_t$ and is thus purely a
measure of the object's flux (in the wavelength range
$\lambda_{\mathrm{min}} <\lambda < \lambda_{\mathrm{max}}$), while the
parameters $a_i/a_0$ (where $i>0$) primarily measure the shape of the
galaxy SED in the wavelength range $\lambda_{\mathrm{min}} <\lambda <
\lambda_{\mathrm{max}}$. $F_t$ can be trivially related to $L_t$, the
total luminosity in this range, by the inverse square of the
luminosity distance (from, for example, \citealt{hogg99a}).  As it
turns out, the parameters $a_i/a_0$ tend to be distributed
approximately in an ellipsoid, with not much curvature within the
space defined by the template SEDs. Thus, we rotate the axes
$\vv{e}_i$ (for $i>0$) such that they are aligned with the principal
axes of the ellipsoid. Furthermore, we normalize the vectors
$\vv{e}_i$ such that the zeroth component $a_0=F_t$ is output by the
code in units of ergs cm$^{-2}$ s$^{-1}$. This choice of coordinate
system is arbitrary and does not affect our quantitative results, but
it is convenient.

In this way, we can reconstruct SEDs based on the broad-band
magnitudes of galaxies. From these reconstructed SEDs one can estimate
$K$-corrections, develop a measure of galaxy spectral type, or
synthesize other measurements of galaxy flux.  The templates
determined during the procedure can, of course, be used to estimate
photometric redshifts, as \citet{csabai00a} and \citet{budavari00a}
describe.

\subsection{Calculating $K$-corrections}

We here define a notation which we will use in subsequent papers for
the effective rest-frame bandpass at any redshift in band $b$, namely:
\begin{equation}
\band{z}{b}
\end{equation}
read, ``$b$-band at redshift $z$.'' The goal of ``$K$-corrections'' is
to convert the $ugriz$ bands at redshift $z$ for each galaxy to
$ugriz$ bands at some fixed redshift $z_0$.

Given AB galaxy magnitudes $^{z}m_{AB}$ and errors for a galaxy at
redshift $z$, one can use the method of the previous subsection to
reconstruct the galaxy rest-frame SED.  With this reconstruction in
hand, it is straightforward to project the SED onto bands at any
redshift $z_0$ ({\it cf.} Equation \ref{ABdef}):
\begin{eqnarray}
^{z_0}m_{\mathrm{rec}} &=& -2.41 - 2.5 \log_{10}\left[
\frac{\int_{0}^{\infty} d\lambda \frac{\lambda}{1+z_0}
f_{\mathrm{rec}}(\frac{\lambda}{1+z_0}) R(\lambda)}
{\int_{0}^{\infty} d\lambda \lambda^{-1} R(\lambda)}\right] \cr
&=& -48.60 - 2.5 \log_{10}\left[
\frac{\int_{0}^{\infty} d\nu \nu^{-1} (1+z_0)
f_{\mathrm{rec}}[\nu (1+z_0)] R(\nu)}
{\int_{0}^{\infty} d\nu \nu^{-1} R(\nu)}\right] 
\end{eqnarray}
The simple way to think about this equation is that simply moving away
from the Earth doing does not change the bolometric flux of the
galaxy. So when the spectrum is stretched by the $(1+z_0)$ redshift
factor, as in $f(\lambda/(1+z_0))$, the bolometric flux should be
constant; that is, you must apply the extra multiplicative factor
$1/(1+z_0)$ to counteract the stretching of the spectrum. 

Given the reconstructed magnitudes, one can calculate the fixed-frame
magnitudes in one of two ways:
\begin{enumerate}
\item Simply use $\band{z_0}{m_{\mathrm{rec}}}$, the reconstructed
magnitudes at redshift $z_0$, as the fixed-frame magnitudes.
\item Use the difference
$\band{z_0}{m_{\mathrm{rec}}}-\band{z}{m_{\mathrm{rec}}}$ as an
estimate of the correction to apply to the observed $\band{z}{m}$.
\end{enumerate}
If the reconstructions $\band{z}{m_{\mathrm{rec}}}$ are very similar
to the actual observations $\band{z}{m}$ (as we will show is the case
for our fits), there is very little difference between these two
methods. Having estimated $\band{z_0}{m}$, you can then calculate the
luminosity by simply applying to the calculated magnitude the
cosmological distance modulus (that is, the luminosity
distance-squared law, as tabulated by, {\it e.g.}  \citealt{hogg97a}).

We want to emphasize here that, while $K$-correcting to a fixed frame
bandpass is {\it sometimes} necessary in order to achieve a scientific
objective, it is not {\it always} necessary or appropriate. Because
$K$-corrections are an inherently uncertain business (the broad-band
magnitudes just do not uniquely determine the SED) they should be
avoid or minimized when possible. For example, if one was calculating
the evolution of clustering of red and blue galaxies separately, it
would perhaps be wise to perform the separation not on the
$K$-corrected colors of the galaxies but on the median color of, say,
$M_\ast$ galaxies as a function of redshift. Similarly, in situation
where $K$-corrections cannot be ignored, such as the calculation of
the evolution of the luminosity function, their effect should be
minimized by, for example, correcting to the median redshift of
galaxies in the sample.

Now, typically one's measurements in will be difficult to connect to,
say, solar luminosities in $\band{z}{b}$ (unless $z=0$), simply
because people have not projected the appropriate stellar
spectrophotometry onto these bandpasses. However, it is our position
that stars are better understood than galaxies, and that it therefore
is in the end simpler to stay as close as possible to the system in
which the galaxies are observed. In any case, astronomy is quickly
reaching a level of precision for which the exact nature of the
bandpasses used has to be known and considered in most analyses of
observational data.

The reader may ask why, if we are basing our $K$-corrections on a full
model of the galaxy SED, we do not correct to bandpasses with simpler
shapes (say, Gaussians). The reason is that we want the effective
bandpasses to actually have been observed for some galaxies in the
sample, so that the $K$-correction for those galaxies is formally and
actually zero. This property is desirable because, as noted above,
there are uncertainties in the $K$-corrections. Nevertheless, we note
that synthesizing such magnitudes may be appropriate in certain
situations, having the virtue (similarly to the $AB$ magnitude system)
that such magnitudes are very easy to comprehend and synthesize from
theory. For example, if we wanted to convert our results into a form
easily comprehended by the future astronomers discussed in the first
paragraph of Section \ref{motivation}, it would be best to calculate
fluxes within wide top-hat filters so that the difficulties of
understanding the magnitude system were minimized.

\subsection{Public Access to the Code}

The version of the $K$-correction code ({\tt v1\_2}) implementing the
calculations described here, along with eigentemplates and filter
curves is publicly available on the World Wide Web at {\tt
http://physics.nyu.edu/\~\ mb144/kcorrect}. In addition, a public CVS
repository can be accessed at {\tt xxx} (instructions for exporting
the appropriate version of the code from this repository are provided
on the web page noted above). The whole of the code can be used
through the Research Systems, Incorporated, IDL language; everything
except for the template-fitting also exists in stand-alone C programs
(which call the same routines, guaranteeing consistency).

\section{Application to SDSS Data}
\label{data}

In this section we describe how we applied the above method to the
SDSS data set.

\subsection{SDSS Spectroscopic Data}

The SDSS (\citealt{york00a}) is producing imaging and spectroscopic
surveys over $\pi$ steradians in the Northern Galactic Cap. A
dedicated 2.5m telescope (Siegmund {\it et al.}, in preparation) at
Apache Point Observatory, Sunspot, New Mexico, images the sky in five
bands between 3000 and 10000 \AA\ ($u$, $g$, $r$, $i$, $z$;
\citealt{fukugita96a}) using a drift-scanning, mosaic CCD camera
(\citealt{gunn98a}), detecting objects to a flux limit of $r'\sim
22.5$. The ultimate goal is to spectroscopically observe 900,000
galaxies, (down to $r_{\mathrm{lim}}'\approx 17.65$), 100,000 Luminous
Red Galaxies (LRGs; Eisenstein {\it et al.}, in preparation), and
100,000 QSOs (\citealt{fan99a}; Newberg {\it et al.}, in preparation).
This spectroscopic follow up uses two digital spectrographs on the
same telescope as the imaging camera. Many of the details of the
galaxy survey are described in a description of the galaxy target
selection paper (\citealt{strauss02a}). Other aspects of the survey
are described in the Early Data Release (EDR;
\citealt{stoughton01a}). The survey has begun in earnest, and has
obtained about 20\% of its intended data.

As of February 2002, the SDSS had targeted and taken spectra of
approximately XXX galaxies over XXX square degrees. As described
below, only a subset of these galaxies are used here to develop the
templates, though we calculate the SED fits for all of them. The
results of the photometric pipeline for all of these objects were
extracted from the SDSS Operational Database {\bf ref munn?}. The
photometry used here for the bulk of these objects was the same as
that used when the objects were targeted. However, for those objects
which were in the EDR photometric catalog, we used the better
calibrations and photometry from the EDR. The versions of the SDSS
pipeline PHOTO used for the reductions of these data ranged from {\tt
v5\_0} to {\tt v5\_2}. The changes in the treatment of relatively
small galaxies, which accounts for most of our sample, did not change
substantially throughout these versions.

The magnitudes are calibrated in the $AB$ system, meaning that the
magnitudes are designed such that:
\begin{eqnarray}
\label{ABdef}
m_{AB} &=& -2.41 - 2.5 \log_{10}\left[
\frac{\int_{0}^{\infty} d\lambda \lambda f(\lambda) R(\lambda)}
{\int_{0}^{\infty} d\lambda \lambda^{-1} R(\lambda)}\right]
\cr
&=& -48.60 - 2.5 \log_{10}\left[
\frac{\int_{0}^{\infty} d\nu \nu^{-1} f(\nu) R(\nu)}
{\int_{0}^{\infty} d\nu \nu^{-1} R(\nu)}\right],
\end{eqnarray}
where $R(\lambda)$ is the fraction of photons entering the Earth's
atmosphere which are detected as a function of wavelength (a unitless
quantity). In this equation, $f(\lambda)$ is in units of ergs
cm$^{-2}$ s$^{-1}$ \AA$^{-1}$, and $f(\nu)$ is in units of ergs
cm$^{-2}$ s$^{-1}$ Hz$^{-1}$. The normalizations defined here mean
that an object with $f(\nu) = 3.631 \times 10^{-20}$ ergs cm$^{-2}$
s$^{-1}$ Hz$^{-1}$ has all its AB magnitudes equal to zero.  The
difference in the zeropoint in the two equations simply corresponds to
the factor of the speed of light $c$ (expressed in \AA\ s$^{-1}$)
between the frequency and wavelength scales.

The response functions $R(\lambda)$ have been measured using a
monochrometer by {\bf how ref Doi}.  Using a model for the atmospheric
transmission and the reflectivity of the primary and secondary
mirrors, one can then model the response of the entire system. For
each bandpass in the SDSS, there are six different CCDs; it has been
shown that the differences between these six camera columns are
small. The resulting set of filter curves is shown in Figure
\ref{response_sdss}, in comparison to a model of a galaxy spectral
energy distribution observed at $z=0$ (a 4 Gyr old instantaneous burst
from the models of \citealt{bruzual93a}).

We extract one-dimensional spectra from the two-dimension images using
a pipeline created specifically for the SDSS instrumentation
(\citealt{schlegel02a}), which also fits for the redshift of each
spectrum. The official SDSS redshifts are obtained from a different
pipeline (\citealt{subbarao02a}). The two independent versions provide
a consistency check on the redshift determination. They are consistent
(for galaxies) at over the 99\% level.\footnote{The official pipeline
tends to work better for objects with unusual spectra, such as certain
types of stars and QSOs.}.

We use two types of magnitudes determined by the SDSS. First, the
modified form of the magnitude described by \citet{petrosian76a}, as
described in \citet{strauss02a}.  These magnitudes have a nearly
constant metric aperture as a function redshift; in addition, all the
bands use the same aperture, so the measured SED corresponds (to
within the effects of seeing) to the SED of an identifiable region of
the galaxy. However, for faint objects, the Petrosian magnitudes tend
to become noisy. Thus, for galaxies with $z>0.25$ we instead use the
higher signal-to-noise ``model magnitudes.'' Model magnitudes are
calculated in all bands using a single weighted aperture. The weighted
aperture is the better fitting model (pure exponential or pure de
Vaucouleurs) to the galaxy image in the $r$-band. In
\citet{stoughton01a} we are explicitly warned (in paragraphs written
by the lead author of this paper!)  not to mix Petrosian and model
magnitudes in a single analysis, since they measure galaxies in very
different ways. Nevertheless, for the purposes of contraining galaxy
template SEDs, it is perfectly acceptable to use {\it any}
well-defined part of any galaxy.  For our purposes, it is more
important to have high signal-to-noise measurement for any
well-defined part of our galaxies, even if the region of the galaxies
probed and their total estimated fluxes are inconsistent with
redshift. We extinction-correct both types of magnitudes using the
dust maps of \citet{schlegel98a}.

\subsection{Fitting SEDs to SDSS Data}

For the purposes of applying the method described in Section
\ref{sedfit} to SDSS data, we take a subsample of the data consisting
of around 25,000 objects in the range $0.0<z<0.5$. The sample includes
both the main sample and the LRGs; we sample these spectra such that
we get an approximately even distribution within our range of
redshifts. In addition, we add results from galaxies in several plates
(totalling about 1,000 objects) which were selected by a photometric
redshift algorithm ({\bf ref who?}) to be at around $z\sim
0.3$--$0.4$. These objects are invaluable for tying down the blue end
of the templates. Finally, we exclude galaxies in the redshift range
$0.28 <z<0.32$ from the fit for the templates, for reasons which we
will explain more fully below (nevertheless, we still can and do use
the resulting templates to analyze galaxies in this range).

Some of the objects have missing or poorly constrained data. For
example, the $u$- or $z$-band fluxes for some objects are swamped by
the photon noise of the sky. We identify such cases as magnitude
errors greater than 0.8 or magnitudes fainter than 22.5 in any
band. We ignore these objects entirely when fitting for the
templates. However, we still want to determine a best-fit SED for each
object. For this purpose, we replace a ``bad'' magnitude based on an
adjacent (usually the redder) band and the average color in those
bands of objects at that redshift. We give these measurements a large
error bar, of 0.8 magnitudes. Thus, we account for the missing
information by simply requiring that the object SED have
``reasonable'' properties.

For the SED space (our $\vv{v}_i(\lambda)$), we use the subspace
defined by linear combinations of ten Bruzual-Charlot instantaneous
burst models with ages ranging from $3 \times 10^7$ to $2\times
10^{10}$ years, five with metallicity $Z=0.02$ and five with
$Z=0.004$, all assuming a Salpeter Initial Mass Function. We find that
we cannot recover information about the SED below a certain scale,
except perhaps around the 4000\AA\ break. Thus, we smooth most of the
wavelength regime of the templates using a Gaussian with a standard
deviation of $\sigma = 300$\AA; around 4000\AA, we smooth only with
$\sigma = 150$\AA.

We choose to fit for $N_t = 4$ eigentemplates, the maximum one can use
and still allow freedom to fit for the templates themselves. Since
galaxies live in a low-dimensional space, we find it is worth being
able to fit for the templates. Simply using five templates (which
obviously reproduces all the magnitudes exactly) tends to yield
unphysical trends of galaxy SED versus redshift ({\it cf.}  Figure
\ref{constantz} below). Using three templates does nearly as well as
four templates in the sense that the resulting templates reproduce the
$griz$ magnitudes nearly as well. However, the fourth template is
necessary to recover the $u$-band flux to better than about 15\%. In
addition, since one of the applications of these SED determinations is
the distribution of galaxy colors in fixed-frame magnitudes, we don't
want to artificially reduce the dimensionality of the color space to
only two.

One more choice needs to be made, the wavelength regime over which to
orthogonalize the templates and over which to calculate the flux. We
choose the range defined by $\lambda_{\mathrm{min}}=3500$\AA\ and
$\lambda_{\mathrm{max}}=7500$\AA, since this range is included for
almost all galaxies in the sample. We refer here and in other papers
to the flux and luminosity in this range as the ``visual flux'' $f_v$
and the ``visual luminosity'' $L_v$.

\subsection{Results}

%The resulting four templates are shown in Figure \ref{k_espec_plot} in
%the range $2000<\lambda<12000$\AA, the region which is constrained by
%our observations. 
The top two panels of Figure \ref{k_coeffdist_plot} show the pairwise
joint distributions of $a_1/a_0$, $a_2/a_0$, and $a_3/a_0$. In the
bottom panel, three spectra taken from a one-dimensional sequence
along the galay locus are shown, showing that the spectra become
progressively bluer along that sequence. Note that constraints on the
SED become poor at the bluest and reddest edges of this diagram (the
peak of the $z$-band response is at 8700\AA). Therefore, the odd
behavior near the edges should not be taken too seriously.

These reconstructed spectra do an excellent job of reproducing the
observed galaxy fluxes. Figure \ref{k_model_plot} shows the
differences between the observed and reconstructed fluxes as a
function of redshift. There are no systematic trends with redshift,
and the standard deviations of the differences between the observed
and the reconstructed fluxes (shown on the right bottom corner of each
panel) are of order the photometric errors in the sample. This
agreement should not be overly surprising given that we have five
galaxy fluxes and four templates to fit, and when one considers the
fact that galaxy colors are known to be correlated with one another.

An important test of the reasonableness of the fits is to check that
for a fixed type of galaxy, the distribution of the fixed frame colors
depends little on redshift. Because the galaxies shown in Figures
\ref{k_coeffdist_plot} and \ref{k_model_plot} are inhomogeneously
selected --- some are Main Sample galaxies, some are Luminous Red
Galaxies, and some are selected from the photometric redshift plates
--- we will split our sample into two well-defined sets of
objects. First, we choose a set of Main Sample galaxies in the
luminosity range $-21.5 < M_{\band{0.1}{r}} < -21$. We reconstruct the
four colors $^{0.1}(u-g)$, $^{0.1}(g-r)$, $^{0.1}(r-i)$, and
$^{0.1}(i-z)$ for these galaxies and plot the reconstructions as a
function of redshift in left-hand panels of Figure
\ref{main_colors_plot}. The plots show a general consistency in the
fixed-frame color distribution of these objects with redshift. The
right-hand panels show the distribution for redshifts $z<0.13$ (solid
histogram) and $z>0.13$ (dotted histogram) . Small changes are
discernible in the distributions, mostly attributable to the increased
errors at higher redshift. Note that for $z<0.1$ the $^{0.1}(u-g)$
color depends on an extrapolation of the SED in the blue, while for
$z>0.1$ the $^{0.1}(i-z)$ color depends on an extrapolation of the SED
in the red.

Second, we choose a set of Cut I LRGs with $-22.8 < M_{\band{0.1}{r}}
< -22.5$. Now we reconstruct the colors to a fixed-frame corresponding
to the rest-frame bands used to observe a galaxy at $z=0.3$.  Near
$z=0.3$ there is some singular behavior of the colors, which we will
discuss in more detail later. Again, for $z<0.3$ the $^{0.3}(u-g)$
color relies on an extrapolation, which is probably responsible for
the scatter in that color at low redshifts. Otherwise, the
distribution of colors is remarkably constant. $^{0.3}(g-r)$ (which is
very similar to $^{0.0}(u-g)$) experiences a blueward shift of about
0.1 magnitudes between $z=0.1$ and $z=0.45$, which is attributable to
passive evolution.

In short, these fits to galaxy SEDs provide estimates which reproduce
the galaxy photometry nearly to the level of the errors in the
photometry itself, seem physically reasonable, and are consistent over
the range of redshifts we consider ($0.0 < z < 0.5$). The only
remaining problem is the aforementioned difficulties at $z=0.3$,
discussed in more detail in the next section.

\section{The $g$/$r$ Gap}
\label{grgap}

It is clear from Figure \ref{k_kcorrect_plot} that $z=0.3$ is a
special redshift regime where our fits become unphysical. In fact, it
is easy to demonstrate that the diagonal spur (going from the upper
left to the lower right) around the red dot in the upper right panel
of Figure \ref{k_coeffdist_plot} is due to galaxies at $z\sim
0.3$. The difficulty arises because the 4000 \AA\ break lies in the
gap between $g$ and $r$ at $z=0.3$. The algorithm requires strong
gradients in the spectrum in this area to explain the 4000\AA break
--- however, the gap in the bands causes a degeneracy arise. 

To demonstrate this degeneracy, Figure \ref{spur} shows the spectrum
corresponding to the direction of the spur in the $a_1/a_0$-$a_3/a_0$
plane of Figure \ref{k_coeffdist_plot}. We overplot the \band{0.3}{g}
and \band{0.3}{r} bands. The spur has a strong peak in the gap between
the two bands at $z=0.3$. Because of this gap, at $z\sim 0.3$ the
algorithm tends to use this direction to help fit the $u$, $i$, and
$z$ fluxes, with relatively little harm done to the $g$ and $r$
fits. This explains why the reconstructed fluxes in Figure
\ref{k_model_plot} fit well even at $z=0.3$, even though the spur at
$z=0.3$ in the distribution of coefficients is so evident.  This
difficulty is why, in Section \ref{data}, we exclude the regime around
$z\sim 0.3$; otherwise the templates became distracted by the
degeneracies in this regime.

The question arises of what to do about this problem. One solution is
to simply work around it. It is clear we cannot use the fits presented
here to $K$-correct galaxies which are at $z=0.3$ to other redshifts
reliably. However, nothing prevents one from $K$-correcting galaxies
at other redshifts {\it to} $z=0.3$. If you are considering the LRG
sample, the median redshift is close to $z=0.3$ anyway, so this is a
viable solution (and the one adopted in Figure \ref{lrg_colors_plot}).

A second solution, not adopted here, is to place stronger constraints
in the SED fit on the form of the 4000 \AA\ break. One obvious way of
doing this is to use the spectra to constrain the size of the break
(making it a weak constraint to prevent biases). This constraint could
be applied in a linear fashion to retain the speed of our method ---
except for the fact that you would have to read in the spectra to
apply the $K$-correction. For simplicity, and because the aforemention
work-around is so easy to obey, we have not implemented this solution
here.

\section{$K$-corrections}
\label{kcorrection}

\subsection{Testing the $K$-corrections}

We show, in Figure \ref{kcorrect}, the $K$-corrections to $z=0.1$
inferred from this method, for all five SDSS bands. Note that the
$K$-corrections are largest (and thus most uncertain) in
$\band{0.1}{u}$ and $\band{0.1}{g}$. 

To test how robust these $K$-correction results are to our model
assumptions, we compare them to other methods of calculating
fixed-frame galaxy magnitudes.  An extremely simple method is to
calculate the flux in any desired bandpass by fitting a power-law
slope and amplitude to the fluxes in the two adjacent bandpasses
(extrapolating when necessary). Figure \ref{ciCompare.sample8b15}
shows the differences in the $K$-corrections inferred from this method
and those inferred from the method of Figure
\ref{kcorrect.sample8b15}, as a function of redshift. $r$, $i$, and
$z$ are all reasonably similar in either method; $u$ and $g$, however,
have distinct trends with redshift, mostly due to the sharper
gradients in the spectra in this regime. To show this fact, we perform
a similar power-law fit, only this time including a break in the
spectrum at 4000\AA. We use the $u$-$g$ color to fit the break,
assuming that the slope blueward of 4000\AA\ is $f(\lambda)\propto
\lambda^{2}$.  Figure \ref{cibreakCompare.sample8b15} shows the
results of this fit; the redshift trend in $g$ is greatly reduced, as
is the trend in $u$, but a large amount of scatter remains in the $u$
band. This results from the fact that you can {\it either} fit the
slope of the SED below the 4000 \AA\ break {\it or} the size of the
4000 \AA\ break itself; it is not possible with this data to constrain
both in an individual spectrum, which is a fundamental limitation of
our analysis of a fixed frame $u$-band luminosity function.

Finally, it is possible to use the galaxy spectra obtained with the
spectrograph to estimate the $K$-correction for each object (in
\band{0.1}{g}, \band{0.1}{r}, and \band{0.1}{i}). However, this
procedure requires trusting the spectrophotometry over a wide
wavelength range. In addition, the spectra only probe the inner parts
of galaxies, and because the fiber aperture is fixed in angular space,
a different region of the galaxy is probed as a function of
redshift. Nevertheless, in Figure \ref{specK}, we compare the
$K$-corrections calculated based on the spectra to those of Figure
\ref{kcorrect}, finding that they are extremely similar. The fact that
they are so similar might prove useful, in the sense that one could
combine the photometric and spectroscopic results, using the spectra
to constrain small-scale features such as the 4000 \AA\ break and the
photometry to cover a large wavelength range and to probe the full
extent of the galaxy. Such a procedure might resolve the difficulties
in handling the $u$-band discussed in the previous paragraph. We will
not pursue this issue further in this paper and leave it for future
investigation.

\section{Conclusions and Future Work}
\label{conclusions}

We have presented a method and an implementation for estimating galaxy
SEDs for the purpose of calculating fixed-frame galaxy magnitudes over
a range of redshifts. We have demonstrated that it gives sensible and
consistent results. We will be using this method in future papers
which will describe the joint distribution of luminosities and colors
of galaxies, as well as the evolution of the luminosity function of
galaxies. Furthermore, we plan to incorporate observations of objects
in bands other than the SDSS bands to further describe the nature of
galaxy SEDs.

Some of the scatter in the three-dimension space describing the shape
of the galaxy SED is probably due to dust. It may be possible to
evaluate the effects of reddening in this space and, by assuming that
galaxies corrected for internal reddening live in an even lower
dimensional space, perform a reddening correction. We will be
investigating this question in the near future.

Furthermore, one may want to include the effects of evolution in the
$K$-corrections. One approach to this calculation is to define a
one-dimensional locus, and record which position on this locus is
closest to each galaxy. Then, the one-dimensional locus provides a set
of spectra, for which estimates of evolution can be calculated. This
procedure avoids fitting star-formation histories to a huge number of
different positions in galaxy SED space. Since their will be
degeneracies between age and metallicity in these fits in any case, it
may be best to simplify the problem to one dimension.

The distribution of galaxies in the three-dimensional space of galaxy
SEDs described here may be useful for other purposes, as well. As one
example, which has already been explored in a separate project, one
can calculate photometric redshifts (if you are willing to restrict
your galaxy models to a one-dimensional, though possibly curved, locus
within the three-dimensional space). However, there is a fairly severe
redshift-type degeneracy, in the sense that blue galaxies at high
redshift can be confused with red galaxies at lower redshift. These
problems can be hazardous to someone trying to estimate the evolution
of the luminosity function from a photometric redshift sample (such
studies rarely include these correlated errors in their estimate of
their uncertainties). However, one can make an alternative set of
assumptions by modeling the distribution of the galaxies in the
three-dimensional spectral space as a pair of gaussians. For such a
model, and a model for the position and luminosity evolution of each
gaussian, one can calculate the probability of observing any given
galaxy. Thus, one could use the galaxy photometry to constrain the
luminosity evolution of galaxies in each gaussian, as well as the
evolution of the position of the gaussians. This method has its own
systematic errors and its own restrictive set of assumptions, but
provides an alternative path to calculating the luminosity function
directly from photometric redshifts.

\acknowledgments

MB was supported at the beginning of this work by the DOE and NASA
grant NAG 5-7092 at Fermilab. He is also grateful for the hospitality
of the Department of Physics and Astronomy at the State University of
New York at Stony Brook, who kindly provided computing facilities on
his frequent visits there. 
%MAS acknowledges the support of NSF grant AST-0071091.

The Sloan Digital Sky Survey (SDSS) is a joint project of The
University of Chicago, Fermilab, the Institute for Advanced Study, the
Japan Participation Group, the Johns Hopkins University, the
Max-Planck-Institute for Astronomy, New Mexico State University,
Princeton University, the United States Naval Observatory, and the
University of Washington. Apache Point Observatory, site of the SDSS
telescopes, is operated by the Astrophysical Research Consortium
(ARC).  Funding for the project has been provided by the Alfred
P.~Sloan Foundation, the SDSS member institutions, the National
Aeronautics and Space Administration, the National Science Foundation,
the U.~S.~Department of Energy, Monbusho, and the Max Planck
Society. The SDSS Web site is http://www.sdss.org/.
 
\begin{thebibliography}{DUM}
\bibitem[Bruzual \& Charlot (1993)]{bruzual93a}
Bruzual, A.~G.,~\& Charlot, S.~1993, \apj, {405}, 538
\bibitem[Bud\'avari {\it et al.} (2000)]{budavari00a}
Budav\'ari, T.; Szalay, A. S.; Connolly, A. J.; Csabai, I.; Dickinson,
M.~(2000), \aj, 120, 1588
\bibitem[Csabai {\it et al.}~(2000)]{csabai00a}
Csabai, I., Connolly, A.~J., Szalay, A.~S., \& Budav\'ari,
T.~2000, \aj, 119, 69
\bibitem[Fan (1999)]{fan99a}
Fan, X.~1999, \aj, 117, 2528
\bibitem[Fukugita {\it et al.}~(1996)]{fukugita96a}
% Galaxy colors in various photometric Band Systems
Fukugita, M., Ichikawa, T., Gunn, J.~E., Doi, M., Shimasaku, K., \&
Schneider, D.~P.~1996, \aj, 111, 1748
\bibitem[Gunn {\it et al.}~(1998)]{gunn98a}
Gunn, J.~E., Carr, M.~A., Rockosi, C.~M., Sekiguchi, M., {\it et al.}~1998, \aj, 116, 3040
\bibitem[Oke \& Sandage (1968)]{oke68a}
Oke, J.~B., \& Sandage, A.~1968, \apj, 154, 21
\bibitem[Petrosian (1976)]{petrosian76a}
Petrosian, V.~1976, \apj, 209, L1
\bibitem[Schlegel, Finkbeiner \& Davis (1998)]{schlegel98a}
Schlegel, D.~J., Finkbeiner, D.~P., \& Davis, M.~1998, \apj, 500, 525
\bibitem[Stoughton {\it et al.} (2001)]{stoughton01a}
Stoughton, C., {\it et al.}~2001, in preparation
\bibitem[York {\it et al.}~(2000)]{york00a}
York, D., {\it et al.}~2000, \aj, 120, 1579

\end{thebibliography}

\newpage

\include{tables}

\clearpage

\setcounter{thefigs}{0}

\clearpage
\stepcounter{thefigs}
\begin{figure}
\figurenum{\fignum}
\plotone{response_sdss.ps}
\caption{\label{response_sdss} Estimated filter response for all five
bands in the SDSS, as a function of observed wavelength. A 4 gigayear
old instantaneous burst using the models of \citet{bruzual93a} (and
observed at $z=0$) is shown for reference.}
%at different redshifts. A 4 gigayear old instantaneous burst using the
%models of \citet{bruzual93a} is shown for reference. {\it Top panel:}
%The system consisting of \band{0.0}{u}, \band{0.0}{g}, \band{0.0}{r},
%\band{0.0}{i}, and \band{0.0}{z}.  {\it Bottom panel:} The system
%consisting of \band{0.0}{u}, \band{0.1}{g}, \band{0.1}{r},
%\band{0.1}{i}, and \band{0.2}{z}. We use this second system rather
%than the first because it requires less interpolation to determine
%\band{0.1}{g}, \band{0.1}{r}, and \band{0.1}{i}, and no extrapolation to
%determine \band{0.0}{u} and \band{0.1}{z}. 
\end{figure}

\clearpage
\stepcounter{thefigs}
\begin{figure}
\figurenum{\fignum}
%\plotone{k_espec_plot.ps}
\caption{\label{k_espec_plot} The four derived eigenspectra. Note that
eigenspectra \#1, \#2, and \#3 are constrained to have zero total flux in the
range between 3500\AA and 7500\AA. Eigenspectrum \#0 is not in any
sense the ``average'' spectrum. }
\end{figure}

\clearpage
\stepcounter{thefigs}
\begin{figure}
\figurenum{\fignum}
\plotone{k_model_plot.ps}
\caption{\label{k_model_plot} Reconstructed galaxy fluxes relative to
the observed galaxy fluxes, for all five SDSS bands, shown for a
random subsample consisting of 16,000 of the SDSS galaxies. The
residuals are shown against redshift.  There is no systematic trend
with redshift in any band. The 5-$\sigma$ clipped estimate of the
scatter around the observed fluxes is listed for each band. In $g$,
$r$, and $i$ the scatter is consistent with the expected photometric
errors in the survey. In $z$ and $u$ there is more scatter. In the
case of $z$ this is attributable to Poisson noise in the
observations. In the case of $u$, it is not, but it may simply be that
there are more intrinic differences between galaxies in the $u$-band
than elsewhere.}
\end{figure}

\clearpage
\stepcounter{thefigs}
\begin{figure}
\figurenum{\fignum}
\plotone{k_coeffdist_plot.ps}
\caption{\label{k_coeffdist_plot} {\it Top panels}: Distribution of
the components of the four-parameter fit to the five-band SDSS
photometry for a random subsample consisting of 16,000 of the SDSS
galaxies. $a_0$ is linearly proportional to the flux between $3500\AA$
and $7000\AA$, while $a_1$, $a_2$, and $a_3$ contribute no flux in
this range. Thus, the ratios $a_1/a_0$ and $a_2/a_0$ describe the
spectral type of the galaxy. $a_2/a_0$ is the most variable parameter
and thus is the better separator of galaxy type. {\it Bottom panel}:
At fixed $a_1/a_0$ and $a_2/a_0$, the inferred spectra corresponding
to various values of $a_3/a_0$. Near $a_3/a_0=-0.16$, the spectrum is
similar to that of an elliptical galaxy. For higher values, the
spectrum becomes bluer. }
\end{figure}

\clearpage
\stepcounter{thefigs}
\begin{figure}
\figurenum{\fignum}
\plotone{k_coeff_plot2.ps}
\caption{\label{k_coeff_plot1} }
\end{figure}

\clearpage
\stepcounter{thefigs}
\begin{figure}
\figurenum{\fignum}
\plotone{k_coeff_plot1.ps}
\caption{\label{k_coeff_plot1} }
\end{figure}

\clearpage
\stepcounter{thefigs}
\begin{figure}
\figurenum{\fignum}
%\plotone{kcorrect.sample8b15.ps}
\caption{\label{kcorrect.sample8b15} $K$-corrections as a function of
redshift in all five bands for a random subsample consisting of 8,000
of the SDSS galaxies. The $K$-corrections are largest, and therefore
the most uncertain, for the \band{0.0}{u} and \band{0.1}{g} bands. The
range of $K$-corrections at each redshift reflects the range of galaxy
types at each redshift.}
\end{figure}

\clearpage
\stepcounter{thefigs}
\begin{figure}
\figurenum{\fignum}
%\plotone{ciCompare.sample8b15.ps}
\caption{\label{ciCompare.sample8b15} Difference in the
$K$-corrections in each band between the method used in Figure
\ref{kcorrect.sample8b15} and the method of simply interpolating
between adjacent bands fitting a power-law SED. The differences are
small in \band{0.1}{r}, \band{0.1}{i}, and \band{0.2}{z}, where galaxy
SEDs have simple shapes. There are large differences in \band{0.0}{u}
and \band{0.1}{g}, for which the 4000 \AA\ break is important. Note
particularly the systematic differences in \band{0.1}{g} with
redshift.}
\end{figure}

\clearpage
\stepcounter{thefigs}
\begin{figure}
\figurenum{\fignum}
%\plotone{cibreakCompare.sample8b15.ps}
\caption{\label{cibreakCompare.sample8b15} Same as Figure
\ref{ciCompare.sample8b15}, now comparing the $K$-corrections of Figure
\ref{kcorrect.sample8b15} with the ``interpolation with a break''
method. This method fits a power law between adjacent bands, except at
4000 \AA, where we fit for the size of the 4000\AA\ break (assuming
that $f(\lambda)\propto \lambda^2$ for $\lambda<4000$ \AA). This
greatly improves the agreement in \band{0.1}{g} while making the
disagreement in \band{0.0}{u} only slightly worse. These results
indicate that it important to account for the structure in the blue
region of the spectrum when performing $K$-corrections.}
\end{figure}

\clearpage
\stepcounter{thefigs}
\begin{figure}
\figurenum{\fignum}
%\plotone{specK.ps}
\caption{\label{specK} Similar to Figure \ref{ciCompare.sample8b15},
now comparing the $K$-corrections of Figure \ref{kcorrect.sample8b15}
with $K$-corrections determined from the spectra (which can only be
calculated for the \band{0.1}{g}, \band{0.1}{r}, and \band{0.1}{i}
bands). In all bands, the $K$-corrections are very similar, giving us
confidence in our results. This result is actually remarkable more for
what it says about the high quality of the spectrophotometry in the
SDSS survey.}
\end{figure}

\clearpage
\stepcounter{thefigs}
\begin{figure}
\figurenum{\fignum}
%\plotone{aperturevsz.M.sample8b15.ps}
\caption{\label{aperturevsz.M.sample8b15} }
\end{figure}



\end{document}
