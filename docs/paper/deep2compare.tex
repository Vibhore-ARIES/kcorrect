\documentclass[10pt,preprint]{aastex}

\newcommand{\vv}[1]{{\bf #1}}
\newcommand{\df}{\delta}
\newcommand{\dfft}{{\tilde{\delta}}}
\newcommand{\betaft}{{\tilde{\beta}}}
\newcommand{\erf}{{\mathrm{erf}}}
\newcommand{\erfc}{{\mathrm{erfc}}}
\newcommand{\Step}{{\mathrm{Step}}}
\newcommand{\ee}[1]{\times 10^{#1}}
\newcommand{\avg}[1]{{\langle{#1}\rangle}}
\newcommand{\Avg}[1]{{\left\langle{#1}\right\rangle}}
\def\simless{\mathbin{\lower 3pt\hbox
	{$\,\rlap{\raise 5pt\hbox{$\char'074$}}\mathchar"7218\,$}}} % < or of order
\def\simgreat{\mathbin{\lower 3pt\hbox
	{$\,\rlap{\raise 5pt\hbox{$\char'076$}}\mathchar"7218\,$}}} % > or of order
\newcommand{\iras}{{\sl IRAS\/}}
\newcommand{\petroratio}{{{\mathcal{R}}_P}}
\newcommand{\petroradius}{{{r}_P}}
\newcommand{\petronumber}{{{N}_P}}
\newcommand{\petroratiolim}{{{\mathcal{R}}_{P,\mathrm{lim}}}}
\newcommand{\band}[2]{\ensuremath{^{{#1}}\!{#2}}}

\setlength{\footnotesep}{9.6pt}

\newcounter{thefigs}
\newcommand{\fignum}{\arabic{thefigs}}

\newcounter{thetabs}
\newcommand{\tabnum}{\arabic{thetabs}}

\newcounter{address}

\slugcomment{To be submitted to \aj}

\shortauthors{Blanton {\it et al.} (2000)}
\shorttitle{ Galaxies at low and high redshift}

\begin{document}
 
\title{ Galaxies at low and high redshift: a comparison of SDSS and
DEEP2}

\author{
Michael R. Blanton\altaffilmark{\ref{NYU}} and 
Samuel Roweis
%Tamas Budavari\altaffilmark{\ref{JHU}},
%Andrew J. Connolly\altaffilmark{\ref{Pitt}},
%J.~Brinkmann\altaffilmark{\ref{APO}},
%Istv\'an Csabai\altaffilmark{\ref{JHU}},
%Mamoru Doi\altaffilmark{\ref{Tokyo}},
%Daniel Eisenstein\altaffilmark{\ref{Arizona}},
%Masataka Fukugita\altaffilmark{\ref{CosmicRay},\ref{IAS}},
%James E. Gunn\altaffilmark{\ref{Princeton}},
%David W. Hogg\altaffilmark{\ref{NYU}}, and
%David J. Schlegel\altaffilmark{\ref{Princeton}}
%Julianne Dalcanton\altaffilmark{\ref{UW}},
%Jon Loveday\altaffilmark{\ref{Sussex}},
%Michael A. Strauss\altaffilmark{\ref{Princeton}},
%Mark SubbaRao\altaffilmark{\ref{Chicago}},
%David H. Weinberg\altaffilmark{\ref{Ohio}},
%John E. Anderson, Jr.\altaffilmark{\ref{Fermilab}},
%James Annis\altaffilmark{\ref{Fermilab}},
%Neta A. Bahcall\altaffilmark{\ref{Princeton}},
%Mariangela Bernardi\altaffilmark{\ref{Chicago}},
%Robert J. Brunner\altaffilmark{\ref{Caltech}},
%Scott Burles\altaffilmark{\ref{Fermilab}},
%Larry Carey\altaffilmark{\ref{UW}},
%Francisco J. Castander\altaffilmark{\ref{Chicago}, \ref{Pyrenees}},
%Andrew J. Connolly\altaffilmark{\ref{Pitt}},
%Istv\'an Csabai\altaffilmark{\ref{JHU}},
%Douglas Finkbeiner\altaffilmark{\ref{Berkeley}},
%Scott Friedman\altaffilmark{\ref{JHU}},
%Joshua A. Frieman\altaffilmark{\ref{Fermilab}},
%G. S. Hennessy\altaffilmark{\ref{USNO}},
%Robert B. Hindsley\altaffilmark{\ref{USNO}},
%Takashi Ichikawa\altaffilmark{\ref{Tokyo}},
%\v{Z}eljko Ivezi\'{c}\altaffilmark{\ref{Princeton}},
%Stephen Kent\altaffilmark{\ref{Fermilab}},
%G. R.~Knapp\altaffilmark{\ref{Princeton}},
%D. Q.~Lamb\altaffilmark{\ref{Chicago}},
%R. French Leger\altaffilmark{\ref{UW}},
%Daniel C. Long\altaffilmark{\ref{APO}},
%Robert H. Lupton\altaffilmark{\ref{Princeton}},
%Timothy A.~McKay\altaffilmark{\ref{Michigan}},
%Avery Meiksin\altaffilmark{\ref{Edinburgh}},
%Aronne Merelli\altaffilmark{\ref{Caltech}},
%Jeffrey A. Munn\altaffilmark{\ref{USNO}},
%Vijay Narayanan\altaffilmark{\ref{Princeton}},
%Matt Newcomb\altaffilmark{\ref{CarnegieMellon}},
%R. C. Nichol\altaffilmark{\ref{CarnegieMellon}},
%Sadanori Okamura\altaffilmark{\ref{Tokyo}},
%Russell Owen\altaffilmark{\ref{UW}},
%Jeffrey R.~Pier\altaffilmark{\ref{USNO}},
%Adrian Pope\altaffilmark{\ref{JHU}},
%Marc Postman\altaffilmark{\ref{STScI}},
%Thomas Quinn\altaffilmark{\ref{UW}},
%Constance M. Rockosi\altaffilmark{\ref{Chicago}},
%Donald P. Schneider\altaffilmark{\ref{PennState}}, 
%Kazuhiro Shimasaku\altaffilmark{\ref{Tokyo}},
%Walter A. Siegmund\altaffilmark{\ref{UW}},
%Stephen Smee\altaffilmark{\ref{Maryland}},
%Yehuda Snir\altaffilmark{\ref{CarnegieMellon}},
%Chris Stoughton\altaffilmark{\ref{Fermilab}},
%Christopher Stubbs\altaffilmark{\ref{UW}},
%Alexander S.~Szalay\altaffilmark{\ref{JHU}},
%Gyula P.~Szokoly\altaffilmark{\ref{Potsdam}},
%Aniruddha R.~Thakar\altaffilmark{\ref{JHU}},
%Christy Tremonti\altaffilmark{\ref{JHU}},
%Douglas L. Tucker\altaffilmark{\ref{Fermilab}},
%Alan Uomoto\altaffilmark{\ref{JHU}},
%Dan vanden Berk\altaffilmark{\ref{Fermilab}},
%Michael S. Vogeley\altaffilmark{\ref{Drexel}},
%Patrick Waddell\altaffilmark{\ref{UW}},
%Brian Yanny\altaffilmark{\ref{Fermilab}},
%Naoki Yasuda\altaffilmark{\ref{NAOJ}},
%and Donald G.~York\altaffilmark{\ref{Chicago}}
}

\altaffiltext{1}{Based on observations obtained with the
Sloan Digital Sky Survey} 
\setcounter{address}{2}
\altaffiltext{\theaddress}{
\stepcounter{address}
New York University, Department of Physics, 4 Washington Place, New
York, NY 10003
\label{NYU}}
%\altaffiltext{\theaddress}{
%\stepcounter{address}
%Department of Physics and Astronomy, The Johns Hopkins University,
%Baltimore, MD 21218
%\label{JHU}}
%\altaffiltext{\theaddress}{
%\stepcounter{address}
%University of Pittsburgh,
%Department of Physics and Astronomy,
%3941 O'Hara Street,
%Pittsburgh, PA 15260
%\label{Pitt}}
%\altaffiltext{\theaddress}{
%\stepcounter{address}
%Department of Astronomy and Research Center for 
%the Early Universe,
%School of Science, University of Tokyo,
%Tokyo 113-0033, Japan
%\label{Tokyo}}
%\altaffiltext{\theaddress}{
%\stepcounter{address}
%Steward Observatory, 
%933 N. Cherry Ave., Tucson, AZ
%85721
%\label{Arizona}}
%\altaffiltext{\theaddress}{
%\stepcounter{address}
%Princeton University Observatory, Princeton,
%NJ 08544
%\label{Princeton}}
%\addtocounter{address}{1}
%\altaffiltext{\theaddress}{
%\stepcounter{address}
%Fermi National Accelerator Laboratory, P.O. Box 500,
%Batavia, IL 60510
%\label{Fermilab}}
%\altaffiltext{\theaddress}{
%\stepcounter{address}
%Department of Astronomy, University of Washington,
%Box 351580,
%Seattle, WA 98195 
%\label{UW}}
%\altaffiltext{\theaddress}{
%\stepcounter{address}
%University of Chicago, Astronomy \&
%Astrophysics Center, 5640 S. Ellis Ave., Chicago, IL 60637
%\label{Chicago}}
%\altaffiltext{\theaddress}{
%\stepcounter{address}
%Hubble Fellow 
%\label{Hubble}}
%\altaffiltext{\theaddress}{
%\stepcounter{address}
%Sussex Astronomy Centre,
%University of Sussex,
%Falmer, Brighton BN1 9QJ, UK
%\label{Sussex}}
%\altaffiltext{\theaddress}{
%\stepcounter{address}
%Ohio State University,
%Department of Astronomy,
%Columbus, OH 43210
%\label{Ohio}}
%\altaffiltext{\theaddress}{
%\stepcounter{address}
%Apache Point Observatory,
%2001 Apache Point Road,
%P.O. Box 59, Sunspot, NM 88349-0059
%\label{APO}}
%\altaffiltext{\theaddress}{
%\stepcounter{address}
%Department of Astronomy, California Institute of Technology,
%Pasadena, CA 91125
%\label{Caltech}}
%\altaffiltext{\theaddress}{
%\stepcounter{address}
%Observatoire Midi-Pyr\'en\'ees, 
%14 ave Edouard Belin, Toulouse, F-31400, France
%\label{Pyrenees}}
%\altaffiltext{\theaddress}{
%\stepcounter{address}
%UC Berkeley, Dept. of Astronomy, 601 Campbell Hall, Berkeley, CA  94720-3411
%\label{Berkeley}}
%\altaffiltext{\theaddress}{
%\stepcounter{address}
%Institute for Cosmic Ray Research, University of
%Tokyo, Midori, Tanashi, Tokyo 188-8502, Japan
%\label{CosmicRay}}
%\altaffiltext{\theaddress}{
%\stepcounter{address}
%Institute for Advanced Study, Olden Lane,
%Princeton, NJ 08540
%\label{IAS}}
%\altaffiltext{\theaddress}{
%\stepcounter{address}
%U.S. Naval Observatory,
%3450 Massachusetts Ave., NW,
%Washington, DC  20392-5420
%\label{USNO}}
%\altaffiltext{\theaddress}{
%\stepcounter{address}
%University of Michigan, Department of Physics,
%500 East University, Ann Arbor, MI 48109
%\label{Michigan}}
%\altaffiltext{\theaddress}{
%\stepcounter{address}
%Department of Physics \& Astronomy,
%The University of Edinburgh,
%James Clerk Maxwell Building,
%The King's Buildings,
%Mayfield Road,
%Edinburgh EH9 3JZ, UK
%\label{Edinburgh}}
%\altaffiltext{\theaddress}{
%\stepcounter{address}
%Department of Physics, Carnegie Mellon University, 
%5000 Forbes Avenue, Pittsburgh, PA 15213-3890 
%\label{CarnegieMellon}}
%\altaffiltext{\theaddress}{
%\stepcounter{address}
%Space Telescope Science Institute, Baltimore, MD 21218
%\label{STScI}}
%\altaffiltext{\theaddress}{
%\stepcounter{address}
%Department of Astronomy and Astrophysics,
%The Pennsylvania State University,
%University Park, PA 16802
%\label{PennState}}
%\altaffiltext{\theaddress}{
%\stepcounter{address}
%Department of Astronomy,
%University of Maryland,
%College Park, MD 20742-2421 
%\label{Maryland}}
%\altaffiltext{\theaddress}{
%\stepcounter{address}
%Astrophysikalisches Institut Potsdam,
%An der Sternwarte 16, D-14482 Potsdam, Germany
%\label{Potsdam}}
%\altaffiltext{\theaddress}{
%\stepcounter{address}
%Department of Physics, Drexel University, Philadelphia, PA 19104
%\label{Drexel}}
%\altaffiltext{\theaddress}{
%\stepcounter{address}
%National Astronomical Observatory, Mitaka, Tokyo 181-8588, Japan
%\label{NAOJ}}
%\addtocounter{address}{1}
%\altaffiltext{\theaddress}{Physics Dept., University of California, Davis, CA 95616
%\label{UCDavis}}
%\addtocounter{address}{1}
%\altaffiltext{\theaddress}{IGPP/Lawrence Livermore National Laboratory
%\label{IGPP}}
%\addtocounter{address}{1}
%\altaffiltext{\theaddress}{Department of Astronomy, University of California, Berkeley, C
%A 94720-3411
%\label{Berkeley}}
%\stepcounter{address}
%\altaffiltext{\theaddress}{Remote Sensing Division, Code 7215, Naval
%Research Laboratory, Washington, DC 20375
%\label{NRL}}
%\addtocounter{address}{1}
%\altaffiltext{\theaddress}{U.S. Naval Observatory, Flagstaff Station,
%P.O. Box 1149,
%Flagstaff, AZ  86002-1149
%\label{Flagstaff}}

\clearpage

\begin{abstract}
The galaxy population has changed significantly in the last half of
the lifetime of the Universe. The old, red sequence galaxy population
so prominent today was not very important at redshift $z \sim 1$. The
blue sequence of galaxies was, at that time, considerably bluer than
it is now. In this paper, we demonstrate these changes by comparing
the galaxies at redshift $z\sim 0.1$ in the Sloan Digital Sky Survey
(SDSS) spectroscopic sample to those in the Deep Extragalactic
Evolutionary Probe 2 (DEEP2) sample. The SDSS sample is nearly
complete, so we perform the comparison by predicting what the SDSS
population would look like at the redshifts probed by DEEP2 and
applying the DEEP2 flux and color selection to the sample. Our results
show dramatic changes in the population between these two epochs.
\end{abstract}

\keywords{galaxies: fundamental parameters --- galaxies: photometry
--- galaxies: statistics}

\section{Motivation}
\label{motivation}

The population of galaxies in the Universe is still changing
dramatically with time. Even within the small range $0 < z < 0.3$ we
can detect that galaxies are fading with time
(\citealt{blanton03d}). In clusters

\section{Discussion}
\label{discussion}

\newpage

%\include{tables}

\clearpage

\setcounter{thefigs}{0}

\clearpage
\stepcounter{thefigs}
\begin{figure}
\figurenum{\fignum}
\plotone{k_sfh.ps}
\caption{\label{sfh} The top panel are the star-formation histories of
the 15 bursts we use in our model. The bottom panel shows (for no dust
and for $Z=0.008$) the PEGASE2 spectrum predicted for these bursts. }
\end{figure}

\clearpage
\stepcounter{thefigs}
\begin{figure}
\figurenum{\fignum}
\plotone{gmrz.ps}
\caption{\label{gmrz} }
\end{figure}

\clearpage
\stepcounter{thefigs}
\begin{figure}
\figurenum{\fignum}
\plotone{kfit.ps}
\caption{\label{kfit} The color residuals between the observed values
  and those of the best fit to the nonnegative combination of 225
  templates ($\delta(b-r)\equiv [b-r]_{\mathrm{obs}} -
  [b-r]_{\mathrm{fit}}$). The greyscale is proportional to the
  distribution of color residual at a given redshift. The lines are
  0.16, 0.45, and 0.84 fractional quantiles. Note that we have zoomed
  into $g-r$, $r-i$, and $i-z$ in order to see the residuals
  better. At high redshift many galaxies have abnormally low $u$-band
  fluxes relative to their $r$-band flux. At low redshift there are a
  few galaxies whose near infrared $z$-band magnitudes are
  lower. There is a general offset of nearly 0.1 magnitudes in
  $J-r$. }
\end{figure}


\end{document}
