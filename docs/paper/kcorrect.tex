\documentclass[10pt,preprint]{aastex}

\newcommand{\vv}[1]{{\bf #1}}
\newcommand{\df}{\delta}
\newcommand{\dfft}{{\tilde{\delta}}}
\newcommand{\betaft}{{\tilde{\beta}}}
\newcommand{\erf}{{\mathrm{erf}}}
\newcommand{\erfc}{{\mathrm{erfc}}}
\newcommand{\Step}{{\mathrm{Step}}}
\newcommand{\ee}[1]{\times 10^{#1}}
\newcommand{\avg}[1]{{\langle{#1}\rangle}}
\newcommand{\Avg}[1]{{\left\langle{#1}\right\rangle}}
\def\simless{\mathbin{\lower 3pt\hbox
	{$\,\rlap{\raise 5pt\hbox{$\char'074$}}\mathchar"7218\,$}}} % < or of order
\def\simgreat{\mathbin{\lower 3pt\hbox
	{$\,\rlap{\raise 5pt\hbox{$\char'076$}}\mathchar"7218\,$}}} % > or of order
\newcommand{\iras}{{\sl IRAS\/}}
\newcommand{\petroratio}{{{\mathcal{R}}_P}}
\newcommand{\petroradius}{{{r}_P}}
\newcommand{\petronumber}{{{N}_P}}
\newcommand{\petroratiolim}{{{\mathcal{R}}_{P,\mathrm{lim}}}}
\newcommand{\band}[2]{\ensuremath{^{{#1}}\!{#2}}}

\newcommand{\kversion}{{\tt v1\_11}}

\setlength{\footnotesep}{9.6pt}

\newcounter{thefigs}
\newcommand{\fignum}{\arabic{thefigs}}

\newcounter{thetabs}
\newcommand{\tabnum}{\arabic{thetabs}}

\newcounter{address}

\slugcomment{To be submitted to \aj}

\shortauthors{Blanton {\it et al.} (2000)}
\shorttitle{$K$-corrections and photometric redshifts}

% TODO:
% finish final templates
% fix kphotoz banding
% fit SFHs to final spectra to get stellar masses, mean ages, 
% mean metallicities

\begin{document}
 
\title{$K$-corrections and photometric redshifts in SDSS and 2MASS$^1$}

\author{
Michael R. Blanton\altaffilmark{\ref{NYU}}
%Tamas Budavari\altaffilmark{\ref{JHU}},
%Andrew J. Connolly\altaffilmark{\ref{Pitt}},
%J.~Brinkmann\altaffilmark{\ref{APO}},
%Istv\'an Csabai\altaffilmark{\ref{JHU}},
%Mamoru Doi\altaffilmark{\ref{Tokyo}},
%Daniel Eisenstein\altaffilmark{\ref{Arizona}},
%Masataka Fukugita\altaffilmark{\ref{CosmicRay},\ref{IAS}},
%James E. Gunn\altaffilmark{\ref{Princeton}},
%David W. Hogg\altaffilmark{\ref{NYU}}, and
%David J. Schlegel\altaffilmark{\ref{Princeton}}
%Julianne Dalcanton\altaffilmark{\ref{UW}},
%Jon Loveday\altaffilmark{\ref{Sussex}},
%Michael A. Strauss\altaffilmark{\ref{Princeton}},
%Mark SubbaRao\altaffilmark{\ref{Chicago}},
%David H. Weinberg\altaffilmark{\ref{Ohio}},
%John E. Anderson, Jr.\altaffilmark{\ref{Fermilab}},
%James Annis\altaffilmark{\ref{Fermilab}},
%Neta A. Bahcall\altaffilmark{\ref{Princeton}},
%Mariangela Bernardi\altaffilmark{\ref{Chicago}},
%Robert J. Brunner\altaffilmark{\ref{Caltech}},
%Scott Burles\altaffilmark{\ref{Fermilab}},
%Larry Carey\altaffilmark{\ref{UW}},
%Francisco J. Castander\altaffilmark{\ref{Chicago}, \ref{Pyrenees}},
%Andrew J. Connolly\altaffilmark{\ref{Pitt}},
%Istv\'an Csabai\altaffilmark{\ref{JHU}},
%Douglas Finkbeiner\altaffilmark{\ref{Berkeley}},
%Scott Friedman\altaffilmark{\ref{JHU}},
%Joshua A. Frieman\altaffilmark{\ref{Fermilab}},
%G. S. Hennessy\altaffilmark{\ref{USNO}},
%Robert B. Hindsley\altaffilmark{\ref{USNO}},
%Takashi Ichikawa\altaffilmark{\ref{Tokyo}},
%\v{Z}eljko Ivezi\'{c}\altaffilmark{\ref{Princeton}},
%Stephen Kent\altaffilmark{\ref{Fermilab}},
%G. R.~Knapp\altaffilmark{\ref{Princeton}},
%D. Q.~Lamb\altaffilmark{\ref{Chicago}},
%R. French Leger\altaffilmark{\ref{UW}},
%Daniel C. Long\altaffilmark{\ref{APO}},
%Robert H. Lupton\altaffilmark{\ref{Princeton}},
%Timothy A.~McKay\altaffilmark{\ref{Michigan}},
%Avery Meiksin\altaffilmark{\ref{Edinburgh}},
%Aronne Merelli\altaffilmark{\ref{Caltech}},
%Jeffrey A. Munn\altaffilmark{\ref{USNO}},
%Vijay Narayanan\altaffilmark{\ref{Princeton}},
%Matt Newcomb\altaffilmark{\ref{CarnegieMellon}},
%R. C. Nichol\altaffilmark{\ref{CarnegieMellon}},
%Sadanori Okamura\altaffilmark{\ref{Tokyo}},
%Russell Owen\altaffilmark{\ref{UW}},
%Jeffrey R.~Pier\altaffilmark{\ref{USNO}},
%Adrian Pope\altaffilmark{\ref{JHU}},
%Marc Postman\altaffilmark{\ref{STScI}},
%Thomas Quinn\altaffilmark{\ref{UW}},
%Constance M. Rockosi\altaffilmark{\ref{Chicago}},
%Donald P. Schneider\altaffilmark{\ref{PennState}}, 
%Kazuhiro Shimasaku\altaffilmark{\ref{Tokyo}},
%Walter A. Siegmund\altaffilmark{\ref{UW}},
%Stephen Smee\altaffilmark{\ref{Maryland}},
%Yehuda Snir\altaffilmark{\ref{CarnegieMellon}},
%Chris Stoughton\altaffilmark{\ref{Fermilab}},
%Christopher Stubbs\altaffilmark{\ref{UW}},
%Alexander S.~Szalay\altaffilmark{\ref{JHU}},
%Gyula P.~Szokoly\altaffilmark{\ref{Potsdam}},
%Aniruddha R.~Thakar\altaffilmark{\ref{JHU}},
%Christy Tremonti\altaffilmark{\ref{JHU}},
%Douglas L. Tucker\altaffilmark{\ref{Fermilab}},
%Alan Uomoto\altaffilmark{\ref{JHU}},
%Dan vanden Berk\altaffilmark{\ref{Fermilab}},
%Michael S. Vogeley\altaffilmark{\ref{Drexel}},
%Patrick Waddell\altaffilmark{\ref{UW}},
%Brian Yanny\altaffilmark{\ref{Fermilab}},
%Naoki Yasuda\altaffilmark{\ref{NAOJ}},
%and Donald G.~York\altaffilmark{\ref{Chicago}}
}

\altaffiltext{1}{Based on observations obtained with the
Sloan Digital Sky Survey} 
\setcounter{address}{2}
\altaffiltext{\theaddress}{
\stepcounter{address}
New York University, Department of Physics, 4 Washington Place, New
York, NY 10003
\label{NYU}}
%\altaffiltext{\theaddress}{
%\stepcounter{address}
%Department of Physics and Astronomy, The Johns Hopkins University,
%Baltimore, MD 21218
%\label{JHU}}
%\altaffiltext{\theaddress}{
%\stepcounter{address}
%University of Pittsburgh,
%Department of Physics and Astronomy,
%3941 O'Hara Street,
%Pittsburgh, PA 15260
%\label{Pitt}}
%\altaffiltext{\theaddress}{
%\stepcounter{address}
%Department of Astronomy and Research Center for 
%the Early Universe,
%School of Science, University of Tokyo,
%Tokyo 113-0033, Japan
%\label{Tokyo}}
%\altaffiltext{\theaddress}{
%\stepcounter{address}
%Steward Observatory, 
%933 N. Cherry Ave., Tucson, AZ
%85721
%\label{Arizona}}
%\altaffiltext{\theaddress}{
%\stepcounter{address}
%Princeton University Observatory, Princeton,
%NJ 08544
%\label{Princeton}}
%\addtocounter{address}{1}
%\altaffiltext{\theaddress}{
%\stepcounter{address}
%Fermi National Accelerator Laboratory, P.O. Box 500,
%Batavia, IL 60510
%\label{Fermilab}}
%\altaffiltext{\theaddress}{
%\stepcounter{address}
%Department of Astronomy, University of Washington,
%Box 351580,
%Seattle, WA 98195 
%\label{UW}}
%\altaffiltext{\theaddress}{
%\stepcounter{address}
%University of Chicago, Astronomy \&
%Astrophysics Center, 5640 S. Ellis Ave., Chicago, IL 60637
%\label{Chicago}}
%\altaffiltext{\theaddress}{
%\stepcounter{address}
%Hubble Fellow 
%\label{Hubble}}
%\altaffiltext{\theaddress}{
%\stepcounter{address}
%Sussex Astronomy Centre,
%University of Sussex,
%Falmer, Brighton BN1 9QJ, UK
%\label{Sussex}}
%\altaffiltext{\theaddress}{
%\stepcounter{address}
%Ohio State University,
%Department of Astronomy,
%Columbus, OH 43210
%\label{Ohio}}
%\altaffiltext{\theaddress}{
%\stepcounter{address}
%Apache Point Observatory,
%2001 Apache Point Road,
%P.O. Box 59, Sunspot, NM 88349-0059
%\label{APO}}
%\altaffiltext{\theaddress}{
%\stepcounter{address}
%Department of Astronomy, California Institute of Technology,
%Pasadena, CA 91125
%\label{Caltech}}
%\altaffiltext{\theaddress}{
%\stepcounter{address}
%Observatoire Midi-Pyr\'en\'ees, 
%14 ave Edouard Belin, Toulouse, F-31400, France
%\label{Pyrenees}}
%\altaffiltext{\theaddress}{
%\stepcounter{address}
%UC Berkeley, Dept. of Astronomy, 601 Campbell Hall, Berkeley, CA  94720-3411
%\label{Berkeley}}
%\altaffiltext{\theaddress}{
%\stepcounter{address}
%Institute for Cosmic Ray Research, University of
%Tokyo, Midori, Tanashi, Tokyo 188-8502, Japan
%\label{CosmicRay}}
%\altaffiltext{\theaddress}{
%\stepcounter{address}
%Institute for Advanced Study, Olden Lane,
%Princeton, NJ 08540
%\label{IAS}}
%\altaffiltext{\theaddress}{
%\stepcounter{address}
%U.S. Naval Observatory,
%3450 Massachusetts Ave., NW,
%Washington, DC  20392-5420
%\label{USNO}}
%\altaffiltext{\theaddress}{
%\stepcounter{address}
%University of Michigan, Department of Physics,
%500 East University, Ann Arbor, MI 48109
%\label{Michigan}}
%\altaffiltext{\theaddress}{
%\stepcounter{address}
%Department of Physics \& Astronomy,
%The University of Edinburgh,
%James Clerk Maxwell Building,
%The King's Buildings,
%Mayfield Road,
%Edinburgh EH9 3JZ, UK
%\label{Edinburgh}}
%\altaffiltext{\theaddress}{
%\stepcounter{address}
%Department of Physics, Carnegie Mellon University, 
%5000 Forbes Avenue, Pittsburgh, PA 15213-3890 
%\label{CarnegieMellon}}
%\altaffiltext{\theaddress}{
%\stepcounter{address}
%Space Telescope Science Institute, Baltimore, MD 21218
%\label{STScI}}
%\altaffiltext{\theaddress}{
%\stepcounter{address}
%Department of Astronomy and Astrophysics,
%The Pennsylvania State University,
%University Park, PA 16802
%\label{PennState}}
%\altaffiltext{\theaddress}{
%\stepcounter{address}
%Department of Astronomy,
%University of Maryland,
%College Park, MD 20742-2421 
%\label{Maryland}}
%\altaffiltext{\theaddress}{
%\stepcounter{address}
%Astrophysikalisches Institut Potsdam,
%An der Sternwarte 16, D-14482 Potsdam, Germany
%\label{Potsdam}}
%\altaffiltext{\theaddress}{
%\stepcounter{address}
%Department of Physics, Drexel University, Philadelphia, PA 19104
%\label{Drexel}}
%\altaffiltext{\theaddress}{
%\stepcounter{address}
%National Astronomical Observatory, Mitaka, Tokyo 181-8588, Japan
%\label{NAOJ}}
%\addtocounter{address}{1}
%\altaffiltext{\theaddress}{Physics Dept., University of California, Davis, CA 95616
%\label{UCDavis}}
%\addtocounter{address}{1}
%\altaffiltext{\theaddress}{IGPP/Lawrence Livermore National Laboratory
%\label{IGPP}}
%\addtocounter{address}{1}
%\altaffiltext{\theaddress}{Department of Astronomy, University of California, Berkeley, C
%A 94720-3411
%\label{Berkeley}}
%\stepcounter{address}
%\altaffiltext{\theaddress}{Remote Sensing Division, Code 7215, Naval
%Research Laboratory, Washington, DC 20375
%\label{NRL}}
%\addtocounter{address}{1}
%\altaffiltext{\theaddress}{U.S. Naval Observatory, Flagstaff Station,
%P.O. Box 1149,
%Flagstaff, AZ  86002-1149
%\label{Flagstaff}}

\clearpage

\begin{abstract}
We present a method to develop templates for calculating
$K$-corrections and photometric redshifts suitable for ultraviolet,
optical, and infrared observations. Since we base the templates on
stellar population synthesis results, the results are intepretable in
terms of approximate stellar masses and star-formation histories. We
present templates fit with this method to the Sloan Digital Sky Survey
and Two-Micron All Sky Survey photometry.  The software that generates
these results is publicly available and easily adapted to handle a
wide range of galaxy observations.
\end{abstract}

\keywords{galaxies: fundamental parameters --- galaxies: photometry
--- galaxies: statistics}

\section{Motivation}
\label{motivation}


\section{Fitting SEDs to broad-band magnitudes}
\label{sedfit}

First, we describe how we reconstruct galaxy SEDs from broad band
magnitude measurements. 

A broad-band flux $F_k$ in filter $k$ is a measurement of the
projection of the source spectrum $f(\lambda)$ on the response
$R_k(\lambda)$ (where typically this function includes the effects of
the atmosphere, telescope, filter $k$, and the CCD itself)
\begin{equation}
F_k = \int_0^\infty d\lambda f(\lambda) R_k(\lambda).
\end{equation}
Now, let us model $f(\lambda)$ as a sum of many template spectra:
\begin{equation}
f(\lambda) = \sum_j a_j v_j(\lambda)
\end{equation}
In order to fit the $a_j$ given the observations in the bands $k$ for
a single object, we minimize
\begin{eqnarray}
\label{chi2}
\chi^2 &=& \sum_{k} \left[ \frac{F_{k,\mathrm{obs}} - F(k)}{\sigma_k}
\right]^2 \cr
&=& \sum_{ik} \left[ \frac{F_{k,\mathrm{obs}} -\sum_j a_j
  v_j(\lambda)}{\sigma_k} \right]^2
\end{eqnarray}
over the $a_j$, where $\sigma_k$ is the standard deviation of the
error in band $k$. Since we will use $v_j(\lambda)$ which are the light
emitted from stars and ionized gas, we make the reasonable assumption
that these only {\it add} light to the spectrum, not subtract it, and
require nonnegativity: that $a_j\ge 0$ for all $j$. (We discuss dust
below).  In order to perform this nonnegative minimization of Equation
\ref{chi2}, we use the method of \citet{sha02a}. 

At this point, the reader probably realizes that the number of
templates $j$ should not be much higher than the number of bands $k$,
or else the problem becomes very degenerate. In fact, for the case of
unconstrained minimization of Equation \ref{chi2} the problem is
formally degenerate if $j>k$. The nonnegativity constraint adds enough
information that this is no longer the case, but still one cannot get
strong constraints on a huge number of parameters given a small number
of measurements. Nevertheless, below we {\it will} use a huge number
of $j$ given the number of $k$, simply for the purposes of finding the
subspace of possible galaxy spectra which actually conform to galaxy
observations. We do not mean to imply that we can determine each of
the resulting parameters at high signal-to-noise.

For our model templates we use a set of spectra from the stellar
evolution synthesis results of PEGASE2 (\citealt{fioc97a}). We use 15
bursts of varying ages. The top panel of Figure \ref{sfh} shows the
star-formation history of each of these bursts. For each age we create
five templates with five different metallicities ($Z=0.03$, $0.02$,
$0.008$, $0.004$, and $0.001$). The bottom panel of Figure \ref{sfh}
shows the template for the $Z=0.008$ case at every other age, starting
with the oldest. The bottom panel of For each age and metallicity, we
create three templates with three different extinctions ($\tau_V=0$,
$1$, and $3$) using the Milky Way extinction law using the ``clumpy,''
``dusty'' model of \citet{witt00a}. This procedure results in 225
templates and spans a very large range of possible spectra. Our
procedure is designed to take this large dimensionality space and
determine a low dimensionality subspace in which galaxies reside.

We fit these models to a set of galaxies for which we know the
redshifts. The majority of the redshifts are from the SDSS, but a
small number of the redshifts are from CNOC2 ({\bf ref}). We consider
only redshifts with 1$^\circ$.25 of the J2000 Equator, between
$-60<\alpha<100$. We match the locations of these spectra to the SDSS
photometric catalog ({\bf describe}) in all of the observations in
this region and to the 2MASS Extended Source Catalog (XSC). In
passing, we note that there are large errors (many arcseconds) in the
catalog astrometry for CNOC2, and we needed to take care to account
for these errors. For spectra which had matches in multiple runs of
the SDSS, we take the (unweighted) mean of their photometry to reduce
errors {\bf how many is this done for?}. We select a random set of
these galaxies in such a way that the number per redshift is nearly
constant across redshift. For each galaxy, we consider its
extinction-corrected model photometry. Figure \ref{gmrz} shows the
observed $g-r$ colors and redshifts of the resulting set of
galaxies. Note that only for the lower redshift galaxies do we have
matches to the 2MASS XSC.

Figure \ref{kfit} shows the residuals between the observed and fit
colors after minimizing Equation \ref{chi2} for all the galaxies in
our sample. Interestingly, even this large set of parameters cannot
perfectly describe the trends of color with redshift. There are
deviations at the percent level or so in $ugriz$. The $J$-band 2MASS
observations at moderate redshift appear to be about $0.1$ magnitudes
off of the models; this discrepancy is probably related to the similar
discrepancies found by \citet{blanton03d} and \citet{baldry03a} when
comparing the luminosity density of the universe to models.  Our
attitude in this paper will be to trust the data in preference to the
models and so below we describe how we deal with these discrepancies.

Each many-parameter fit to the broad-band fluxes of each galaxy yields
a complete spectrum from which to derive $K$-corrections. It also
provides a value of $\chi^2$; in principle, for galaxies without
redshifts we can find the assumed redshift with minimum $\chi^2$ in
order to find the photometric redshift. In practice, the fit takes too
long to perform to be useful for either of these purposes. Thus, we
describe here our method for reducing the dimensionality to a set of
three templates which adequately describe the data.

Our procedure has two steps. First, we use principle component
analysis on the set of 225-coefficent vectors describing all the fits
in order to find a 10-dimensional subspace which still describes the
data. Second, within this 10-dimensional subspace we group the
galaxies into a set of ten groups, average the spectra within each
group, and pick the triple among these groups which best describes the
data. We adopted this procedure because we did not know of a
principled way in which to reduce the dimensionality without
sacrificing nonnegativity; however, it turns out that such methods do
exist, and we recommend those methods for future investigators ({\bf
refs from sam}) while retaining this method for the moment because it
appears to work for our purposes.

Because the galaxies all have very different bolometric fluxes, there
is a strong variation in the coefficient values simply due to the
total flux differing by large factors. Therefore, before performing
principal component analysis, we scale all the coefficients of each
galaxy by a single number (per galaxy) such that the galaxies all lie
exactly on a plane in coefficient space. We choose the plane to be the
plane on which the sum of the templates has a constant flux between
1500 \AA\ and 30000 \AA. After scaling the galaxies to this plane, we
perform principal component analysis to obtain the five most principal
axes, using the efficient method of \citet{roweis99a}. There is almost
no difference between the full fit shown above and that using the
first five eigencoefficients.

On the other hand, the five eigencomponents natural consist partly of
positive star-formation components and partly of negative
star-formation components. This means that we cannot simply sum them
nonnegatively to get a nonnegative star-formation history. In order to
get a set of components that consists purely of all-positive
star-formation histories, we consider the distribution of galaxies in
the five-component space. We perform a $K$-means grouping on this set
of points, which produces ten groups. This is simply a way of very
roughly mapping the locus on which the galaxies reside (it is
significantly nonlinear within the five-dimensional space, so we
cannot simply take fewer principal components). We then average the
spectra in each group. This gives ten spectra who roughly span the
space of galaxy types. Finally, we pick from among those ten the
triple which best describes the galaxies, in the sense of minimizing
$\chi^2$ in Equation \ref{chi2}. 

Figure \ref{kfit3} shows the color residuals after fitting with the
final three templates. 

[ new model: 3 PEGASE2 spectra; Figure show spectra, Figure stellar
  mass to light ratios ]

[ to deal with that for K-correction and photo-z, tweak ]
 
We recognize that this procedure is rather convoluted and a bit of a
hack. As noted above, we have adopted it simply as one way of
automatically getting a small set of templates that explain galaxy
broad-band fluxes as a sum of star-formation histories. There are
methods for directly performing principal component analysis subject
to the conditions that the components and the coefficients are all
positive, and we recommend the investigation of such methods to the
reader ({\bf ref}).

\section{Determining $K$-corrections} 
\label{kcorrect}

Given a model spectrum for the galaxy, the determination of the
$K$-correction is straightforward. Here we repeat the 

[ Figure demonstrate K-corrections ]

[ Figure show K-corrections to z=0 ]

[ note band-shifting ]

[ Figure test: against a bunch of spectra ]

\section{Determining photometric redshifts}
\label{kphotoz}

[ describe simple fitting procedure ]

[ Figure show training set results ]

[ Figure show results on deeper sdss data ] 

\section{Public access to the code}

\section{Summary}
\label{conclusions}

[ summarize ]

[ this does not describe what constraints photometry puts on the SFH ]

\acknowledgments

Sam Roweis

Funding for the creation and distribution of the SDSS Archive has been
provided by the Alfred P. Sloan Foundation, the Participating
Institutions, the National Aeronautics and Space Administration, the
National Science Foundation, the U.S. Department of Energy, the
Japanese Monbukagakusho, and the Max Planck Society. The SDSS Web site
is {\tt http://www.sdss.org/}.

The SDSS is managed by the Astrophysical Research Consortium (ARC) for
the Participating Institutions. The Participating Institutions are The
University of Chicago, Fermilab, the Institute for Advanced Study, the
Japan Participation Group, The Johns Hopkins University, Los Alamos
National Laboratory, the Max-Planck-Institute for Astronomy (MPIA),
the Max-Planck-Institute for Astrophysics (MPA), New Mexico State
University, Princeton University, the United States Naval Observatory,
and the University of Washington.
 
\bibliographystyle{../../../nyu-astro/tex/apj}
\bibliography{../../../nyu-astro/tex/apj-jour,../../../nyu-astro/tex/ccpp}

\appendix

\section{Approximate conversions between common magnitude systems}

\newpage

%\include{tables}

\clearpage

\setcounter{thefigs}{0}

\clearpage
\stepcounter{thefigs}
\begin{figure}
\figurenum{\fignum}
\plotone{k_sfh.ps}
\caption{\label{sfh} The top panel are the star-formation histories of
the 15 bursts we use in our model. The bottom panel shows (for no dust
and for $Z=0.008$) the PEGASE2 spectrum predicted for these bursts. }
\end{figure}

\clearpage
\stepcounter{thefigs}
\begin{figure}
\figurenum{\fignum}
\plotone{gmrz.ps}
\caption{\label{gmrz} }
\end{figure}

\clearpage
\stepcounter{thefigs}
\begin{figure}
\figurenum{\fignum}
\plotone{kfit.ps}
\caption{\label{kfit} The color residuals between the observed values
  and those of the best fit to the nonnegative combination of 225
  templates ($\delta(b-r)\equiv [b-r]_{\mathrm{obs}} -
  [b-r]_{\mathrm{fit}}$). The greyscale is proportional to the
  distribution of color residual at a given redshift. The lines are
  0.16, 0.45, and 0.84 fractional quantiles. Note that we have zoomed
  into $g-r$, $r-i$, and $i-z$ in order to see the residuals
  better. At high redshift many galaxies have abnormally low $u$-band
  fluxes relative to their $r$-band flux. At low redshift there are a
  few galaxies whose near infrared $z$-band magnitudes are
  lower. There is a general offset of nearly 0.1 magnitudes in
  $J-r$. }
\end{figure}


\end{document}
