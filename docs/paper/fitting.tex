\documentclass[10pt]{aastex}

\newcommand{\vv}[1]{{\bf #1}}
\newcommand{\df}{\delta}
\newcommand{\dfft}{{\tilde{\delta}}}
\newcommand{\betaft}{{\tilde{\beta}}}
\newcommand{\erf}{{\mathrm{erf}}}
\newcommand{\erfc}{{\mathrm{erfc}}}
\newcommand{\Step}{{\mathrm{Step}}}
\newcommand{\ee}[1]{\times 10^{#1}}
\newcommand{\avg}[1]{{\langle{#1}\rangle}}
\newcommand{\Avg}[1]{{\left\langle{#1}\right\rangle}}
\def\simless{\mathbin{\lower 3pt\hbox
	{$\,\rlap{\raise 5pt\hbox{$\char'074$}}\mathchar"7218\,$}}} % < or of order
\def\simgreat{\mathbin{\lower 3pt\hbox
	{$\,\rlap{\raise 5pt\hbox{$\char'076$}}\mathchar"7218\,$}}} % > or of order
\newcommand{\iras}{{\sl IRAS\/}}
\newcommand{\petroratio}{{{\mathcal{R}}_P}}
\newcommand{\petroradius}{{{r}_P}}
\newcommand{\petronumber}{{{N}_P}}
\newcommand{\petroratiolim}{{{\mathcal{R}}_{P,\mathrm{lim}}}}
\newcommand{\band}[2]{\ensuremath{^{{#1}}\!{#2}}}

\newcommand{\kversion}{{\tt v1\_11}}

\setlength{\footnotesep}{9.6pt}

\newcounter{thefigs}
\newcommand{\fignum}{\arabic{thefigs}}

\newcounter{thetabs}
\newcommand{\tabnum}{\arabic{thetabs}}

\newcounter{address}

\slugcomment{To be submitted to \aj}

\shortauthors{Blanton {\it et al.} (2005)}
\shorttitle{Template star-formation histories}

\begin{document}

\title{ Using heterogeneous data to constrain star-formation histories
of galaxies}

\author{
Michael R. Blanton\altaffilmark{\ref{NYU}} and 
Sam Roweis\altaffilmark{\ref{NYU}}}

\setcounter{address}{1}
\altaffiltext{\theaddress}{
\stepcounter{address}
New York University, Department of Physics, 4 Washington Place, New
York, NY 10003
\label{NYU}}
\clearpage

\begin{abstract}
Astronomers have a huge set of well-calibrated photometric and
spectroscopic data on galaxies from recent large surveys at all
redshifts, such as (at low redshift) the GALaxy Evolution eXplorer in
the ultraviolet, the Sloan Digital Sky Survey in the optical, and the
Two-Micron All Sky Survey in the near infrared, and (at higher
redshifts) the Deep Extragalactic Evolutionary Probe 2 and the Great
Observatories Origins Deep Survey.  However, this data comes in a
rather heterogeneous form --- sometimes in the form of spectra,
sometimes in the form of broad-band photometry. In addition,
broad-band photometry suffers from two forms of heterogeneity: system
responses differ among different surveys, and galaxies are observed at
many different redshifts (and the system responses therefore differ
among all galaxies in the rest frame).  How can we explore this very
heterogeneous set of data?  Here we present an implementation of an
algorithm for fitting star-formation histories to these data. Our
approach is to restrict the problem by trying to explain the
observations of each galaxy as a nonnegative linear combination of a
small set of star-formation history templates.  The algorithm has
almost arbitrary freedom in choosing the set of templates, but it must
explain every galaxy in terms of a nonnegative sum of the same set.
Our algorithm can handle whatever spectral or broad-band information
one has on each galaxy.  The results of the algorithm have scientific
applications --- such as exploring the star-formation histories of
galaxies, how they vary with environment, how they correlate with
other properties, and how they change with redshift. In addition, they
have technical applications --- such as $K$-corrections, photometric
redshifts, and continuum-fitting of spectra. This algorithm and its
ilk are likely to become more and more useful as astronomers continue
to combine large sets of inhomogeneous data.
\end{abstract}

\keywords{galaxies: fundamental parameters --- galaxies: photometry
--- galaxies: statistics}

\section{Motivation}
\label{motivation}

Astronomers have observed the broad-band colors and spectra of a huge
number of galaxies in the Universe, in many different passbands.  For
example, at low redshift the Galaxy Evolution Explorer (GALEX; {\bf
ref}) observes in the ultraviolet (UV), the Sloan Digital Sky Survey
(SDSS; \citealt{york00a}) observes spectroscopically and
photometrically in the optical, and the Two-Micron All Sky Survey
(2MASS; \citealt{strutskie97a}) observes in the near infrared
(NIR). At higher redshifts, the Deep Extragalactic Evolutionary Probe
2 (DEEP2; {\bf ref} and the Great Observatories Origins Deep Survey
(GOODS; {\bf ref}) observe in the rest-frame UV. These and other data
sets provide a huge set of information about galaxy colors and spectra
which we can use to help understand their star-formation histories.

However, this data comes in rather heterogeneous forms. Some of the
information is spectrophotometric, and some of it is broad-band
photometry. Broad-band photometry differs among different instruments:
for example, the SDSS uses the $ugriz$ bands to observe in the
optical, whereas DEEP2 uses the $BRI$ bands. In addition, these bands
probe different parts of the rest-frame galaxy spectrum at different
redshifts. So the constraints these observations apply to our fits of
the star-formation histories of these objects vary considerably from
one object to another.

On the other hand, from the spectroscopic observations we know the
galaxy spectra reside in a low-dimensional locus. Principal Component
Analysis (PCA) of galaxy spectra, introduced by \citet{connolly94a} and
applied many times since then ({\bf more refs}; \citealt{yip05a}),
demonstrates that most of the variance in the distribution of galaxies
in spectral space can be explained using a few templates. This means,
that in the linear space of all possible spectra, galaxies exist in
only a small local. Therefore, even with very heterogeneous data, we
should be able to determine the properties of this locus.

Here we present an approach to combining the heterogeneous data
described above in order to determine the properties of the locus of
galaxy spectra. Rather than taking the model-free approach used by
PCA, we here restrict the space of possible spectra to those predicted
from the high resolution stellar population synthesis model of
\citet{bruzual03a}. This approach both constrains the problem
appropriately and yields a natural theoretical interpretation of the
results in terms of star-formation histories.

In a nutshell, our algorithm does the following. Given the
observations (and uncertainties of those observations) available for
each galaxy, it finds the nonnegative linear combination of $N$
template star-formation histories which best predict those
observations in the $\chi^2$ sense. Given the entire set of galaxy
observations available, it also fits for those template star-formation
histories. The technical name for this algorithm is Nonnegative Matrix
Factorization (NMF).

This approach is similar to PCA in that it finds the small spectral
subspace in which galaxies exist, and can in some ways be thought of
as ``nonnegative PCA.''  However, our method has several advantages
over standard PCA approach. First, it has a natural physical
interpretation associated with that spectral subspace, which is the
corresponding subspace of all possible star-formation
histories. Second, it naturally handles data uncertainties and missing
data which allows it to ignore variation which is due purely to
statistical errors.  Third, it handles the complications of observing
galaxy spectra photometrically using broad-band filters.

The results of this work yield, first of all, a set of templates which
are useful for any number of technical tasks, for example
$K$-corrections, evolution corrections, photometric redshifts, and
continuum subtraction of galaxy spectra. Second, the fits to
individual galaxies provide a theoretical interpretation of their
emission, in terms of their stellar masses, dust content, and
star-formation histories.

In Section \ref{algorithm} we present and test the NMF algorithm
itself. All of the software we use is available publicly in the {\tt
idlutils}, {\tt idlspec2d} and {\tt kcorrect} packages.  In Section
\ref{data} we present the data we use, most of which is publicly
available. In Section \ref{results}, we present the templates we have
found for our data set. We summarize in Section
\ref{summary}. Separate papers describe the application of these
results to the scientific and technical questions described above.

\section{Method}
\label{algorithm}

Our method is fundamentally very simple. We take $N$ star-formation
history templates and all of our observations of our galaxies.  We
then find the templates and (for each galaxy) their linear combination
by minimizing the $\chi^2$ difference between the actual observations
and the model predictions for those observations. In practice there
are some complications and details to consider. In Section \ref{model}
we describe how we express the star-formation histories of each
galaxy.  In Section \ref{spectra} we explain the models we use to
convert these star-formation histories into spectra. In Section
\ref{observe} we describe how a particular star-formation history can
be transformed into all possible observations for that galaxy. In
Section \ref{nmf} we describe our minimization procedure.

\subsection{Model for the star-formation histories}
\label{model}

Our model for the star-formation history of each galaxy $i$ is a linear
combination of $N$ templates, as follows:
\begin{equation}
G_i(t, m, d) = \sum_{i=0}^{N-1} a_{ik} T_k(t, m, \tau_V).
\end{equation}
We restrict to nonnegative star-formation rates ($a_{ik} \ge
0$). $T_k(t, m, \tau_V)$ is the star-formation history, described as
the stellar mass (in solar masses) formed $t$ years ago, with
metallicity $m$, and currently obscured by $\tau_V$ magnitudes in the
$V$ band.

Each template is in turn expressed as a linear combination of a set of
instantaneous bursts:
\begin{equation}
T_k(t, m, d) = 
\sum_{l=0}^{N_t-1} 
\sum_{n=0}^{N_m-1} 
\sum_{p=0}^{N_\tau-1} 
t_{lmn} 
\delta^D(t-t_l)
\delta^D(m-m_n)
\delta^D(\tau_V-\tau_{V,p})
\end{equation}



\subsection{Predicted spectra from the star-formation histories}
\label{spectra}


Note that we do not {\it have} to limit our spectral library to those
predicted by star-formation histories. In fact, we can add arbitrary
spectra in this step to account for emission lines, polycyclic
aromatic hydrocarbon (PAH) emission, active galactic nuclei, or any
other constituent of galaxies. 

In fact, as an example of this, we here include nine emission lines:
OII 3728 \AA; H$\beta$ 4863 \AA; OIII 4960, 5008 \AA; NII 6550, 6585
\AA; H$\alpha$ 6565 \AA; SII 6718, 6733 (all vacuum wavelengths). We
include each of these lines as a template with a Gaussian centered on
the line center with a dispersion corresponding to 300 km s^{-1}.

\subsection{Observations of the predicted spectra}
\label{observe}

\subsection{Finding the best-fit templates}
\label{nmf}

\section{Data}
\label{data}

\section{Results}
\label{results}

\section{Summary}
\label{summary}


\acknowledgments

Funding for the creation and distribution of the SDSS Archive has been
provided by the Alfred P. Sloan Foundation, the Participating
Institutions, the National Aeronautics and Space Administration, the
National Science Foundation, the U.S. Department of Energy, the
Japanese Monbukagakusho, and the Max Planck Society. The SDSS Web site
is {\tt http://www.sdss.org/}.

The SDSS is managed by the Astrophysical Research Consortium (ARC) for
the Participating Institutions. The Participating Institutions are The
University of Chicago, Fermilab, the Institute for Advanced Study, the
Japan Participation Group, The Johns Hopkins University, Los Alamos
National Laboratory, the Max-Planck-Institute for Astronomy (MPIA),
the Max-Planck-Institute for Astrophysics (MPA), New Mexico State
University, Princeton University, the United States Naval Observatory,
and the University of Washington.
 
\begin{thebibliography}{DUM}
\bibitem[Bruzual \& Charlot (1993)]{bruzual93a}
Bruzual, A.~G.,~\& Charlot, S.~1993, \apj, {405}, 538
\bibitem[Bud\'avari {\it et al.} (2000)]{budavari00a}
Budav\'ari, T.; Szalay, A. S.; Connolly, A. J.; Csabai, I.; Dickinson,
M.~(2000), \aj, 120, 1588
\bibitem[Csabai {\it et al.}~(2000)]{csabai00a}
Csabai, I., Connolly, A.~J., Szalay, A.~S., \& Budav\'ari,
T.~2000, \aj, 119, 69
\bibitem[Fan (1999)]{fan99a}
Fan, X.~1999, \aj, 117, 2528
\bibitem[Fukugita {\it et al.}~(1996)]{fukugita96a}
% Galaxy colors in various photometric Band Systems
Fukugita, M., Ichikawa, T., Gunn, J.~E., Doi, M., Shimasaku, K., \&
Schneider, D.~P.~1996, \aj, 111, 1748
\bibitem[Gunn {\it et al.}~(1998)]{gunn98a}
Gunn, J.~E., Carr, M.~A., Rockosi, C.~M., Sekiguchi, M., {\it et al.}~1998, \aj, 116, 3040
\bibitem[Oke \& Sandage (1968)]{oke68a}
Oke, J.~B., \& Sandage, A.~1968, \apj, 154, 21
\bibitem[Petrosian (1976)]{petrosian76a}
Petrosian, V.~1976, \apj, 209, L1
\bibitem[Schlegel, Finkbeiner \& Davis (1998)]{schlegel98a}
Schlegel, D.~J., Finkbeiner, D.~P., \& Davis, M.~1998, \apj, 500, 525
\bibitem[Stoughton {\it et al.} (2001)]{stoughton01a}
Stoughton, C., {\it et al.}~2001, in preparation
\bibitem[York {\it et al.}~(2000)]{york00a}
York, D., {\it et al.}~2000, \aj, 120, 1579

\end{thebibliography}

\newpage

%\include{tables}

\clearpage

\setcounter{thefigs}{0}

\clearpage
\stepcounter{thefigs}
\begin{figure}
\figurenum{\fignum}
\plotone{response_sdss.ps}
\caption{\label{response_sdss} Estimated filter response for all five
bands in the SDSS, as a function of observed wavelength. A 4 gigayear
old instantaneous burst using the models of \citet{bruzual93a} (and
observed at $z=0$) is shown for reference.}
%at different redshifts. A 4 gigayear old instantaneous burst using the
%models of \citet{bruzual93a} is shown for reference. {\it Top panel:}
%The system consisting of \band{0.0}{u}, \band{0.0}{g}, \band{0.0}{r},
%\band{0.0}{i}, and \band{0.0}{z}.  {\it Bottom panel:} The system
%consisting of \band{0.0}{u}, \band{0.1}{g}, \band{0.1}{r},
%\band{0.1}{i}, and \band{0.2}{z}. We use this second system rather
%than the first because it requires less interpolation to determine
%\band{0.1}{g}, \band{0.1}{r}, and \band{0.1}{i}, and no extrapolation to
%determine \band{0.0}{u} and \band{0.1}{z}. 
\end{figure}

\clearpage
\stepcounter{thefigs}
\begin{figure}
\figurenum{\fignum}
%\plotone{k_espec_plot.ps}
\caption{\label{k_espec_plot} The four derived eigenspectra. Note that
eigenspectra \#1, \#2, and \#3 are constrained to have zero total flux in the
range between 3500\AA and 7500\AA. Eigenspectrum \#0 is not in any
sense the ``average'' spectrum. }
\end{figure}

\clearpage
\stepcounter{thefigs}
\begin{figure}
\figurenum{\fignum}
\plotone{k_model_plot.ps}
\caption{\label{k_model_plot} Reconstructed galaxy fluxes relative to
the observed galaxy fluxes, for all five SDSS bands, shown for a
random subsample consisting of 16,000 of the SDSS galaxies. The
residuals are shown against redshift.  There is no systematic trend
with redshift in any band. The 5-$\sigma$ clipped estimate of the
scatter around the observed fluxes is listed for each band. In $g$,
$r$, and $i$ the scatter is consistent with the expected photometric
errors in the survey. In $z$ and $u$ there is more scatter. In the
case of $z$ this is attributable to Poisson noise in the
observations. In the case of $u$, it is not, but it may simply be that
there are more intrinic differences between galaxies in the $u$-band
than elsewhere.}
\end{figure}

\clearpage
\stepcounter{thefigs}
\begin{figure}
\figurenum{\fignum}
\plotone{k_coeffdist_plot.ps}
\caption{\label{k_coeffdist_plot} {\it Top panels}: Distribution of
the components of the four-parameter fit to the five-band SDSS
photometry for a random subsample consisting of 16,000 of the SDSS
galaxies. $a_0$ is linearly proportional to the flux between $3500\AA$
and $7000\AA$, while $a_1$, $a_2$, and $a_3$ contribute no flux in
this range. Thus, the ratios $a_1/a_0$ and $a_2/a_0$ describe the
spectral type of the galaxy. $a_2/a_0$ is the most variable parameter
and thus is the better separator of galaxy type. {\it Bottom panel}:
At fixed $a_1/a_0$ and $a_2/a_0$, the inferred spectra corresponding
to various values of $a_3/a_0$. Near $a_3/a_0=-0.16$, the spectrum is
similar to that of an elliptical galaxy. For higher values, the
spectrum becomes bluer. }
\end{figure}

\clearpage
\stepcounter{thefigs}
\begin{figure}
\figurenum{\fignum}
\plotone{k_coeff_plot2.ps}
\caption{\label{k_coeff_plot1} }
\end{figure}

\clearpage
\stepcounter{thefigs}
\begin{figure}
\figurenum{\fignum}
\plotone{k_coeff_plot1.ps}
\caption{\label{k_coeff_plot1} }
\end{figure}

\clearpage
\stepcounter{thefigs}
\begin{figure}
\figurenum{\fignum}
%\plotone{kcorrect.sample8b15.ps}
\caption{\label{kcorrect.sample8b15} $K$-corrections as a function of
redshift in all five bands for a random subsample consisting of 8,000
of the SDSS galaxies. The $K$-corrections are largest, and therefore
the most uncertain, for the \band{0.0}{u} and \band{0.1}{g} bands. The
range of $K$-corrections at each redshift reflects the range of galaxy
types at each redshift.}
\end{figure}

\clearpage
\stepcounter{thefigs}
\begin{figure}
\figurenum{\fignum}
%\plotone{ciCompare.sample8b15.ps}
\caption{\label{ciCompare.sample8b15} Difference in the
$K$-corrections in each band between the method used in Figure
\ref{kcorrect.sample8b15} and the method of simply interpolating
between adjacent bands fitting a power-law SED. The differences are
small in \band{0.1}{r}, \band{0.1}{i}, and \band{0.2}{z}, where galaxy
SEDs have simple shapes. There are large differences in \band{0.0}{u}
and \band{0.1}{g}, for which the 4000 \AA\ break is important. Note
particularly the systematic differences in \band{0.1}{g} with
redshift.}
\end{figure}

\clearpage
\stepcounter{thefigs}
\begin{figure}
\figurenum{\fignum}
%\plotone{cibreakCompare.sample8b15.ps}
\caption{\label{cibreakCompare.sample8b15} Same as Figure
\ref{ciCompare.sample8b15}, now comparing the $K$-corrections of Figure
\ref{kcorrect.sample8b15} with the ``interpolation with a break''
method. This method fits a power law between adjacent bands, except at
4000 \AA, where we fit for the size of the 4000\AA\ break (assuming
that $f(\lambda)\propto \lambda^2$ for $\lambda<4000$ \AA). This
greatly improves the agreement in \band{0.1}{g} while making the
disagreement in \band{0.0}{u} only slightly worse. These results
indicate that it important to account for the structure in the blue
region of the spectrum when performing $K$-corrections.}
\end{figure}

\clearpage
\stepcounter{thefigs}
\begin{figure}
\figurenum{\fignum}
%\plotone{specK.ps}
\caption{\label{specK} Similar to Figure \ref{ciCompare.sample8b15},
now comparing the $K$-corrections of Figure \ref{kcorrect.sample8b15}
with $K$-corrections determined from the spectra (which can only be
calculated for the \band{0.1}{g}, \band{0.1}{r}, and \band{0.1}{i}
bands). In all bands, the $K$-corrections are very similar, giving us
confidence in our results. This result is actually remarkable more for
what it says about the high quality of the spectrophotometry in the
SDSS survey.}
\end{figure}

\clearpage
\stepcounter{thefigs}
\begin{figure}
\figurenum{\fignum}
%\plotone{aperturevsz.M.sample8b15.ps}
\caption{\label{aperturevsz.M.sample8b15} }
\end{figure}



\end{document}
